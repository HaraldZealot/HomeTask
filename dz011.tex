\documentclass[12pt]{article} 
\usepackage[xetex, a4paper, left=2cm, right=2cm, top=2cm,bottom=2cm]{geometry}
\usepackage[cm-default]{fontspec}
\usepackage{xunicode}

%\tolerance=1000
%\emergencystretch=0.74cm 

\usepackage{polyglossia}
\setdefaultlanguage[spelling=modern]{russian}
\setotherlanguage{english} 
\defaultfontfeatures{Scale=MatchLowercase,Ligatures=TeX}  %% устанавливает поведение шрифтов по умолчанию  
\newfontfamily\cyrillicfont{Linux Libertine} 
\setromanfont[Mapping=tex-text]{Linux Libertine}
\setsansfont[Mapping=tex-text]{Linux Biolinum}
\setmonofont{DejaVu Sans Mono}
%\newfontfamily\cyrillicfont{Liberation Mono} 

%\usepackage{makecell}

%\usepackage{titlesec}
%\newcommand{\sectionbreak}{\clearpage}

%\renewcommand{\thesection}{\Alph{section}}
%\newcount\wd    \wd=\textwidth \multiply\wd by 8 \divide\wd by 17

\usepackage{minted}
\usemintedstyle{friendly}

\usepackage[unicode, pdfborder={0 0 0 0}]{hyperref}

\author{Alaksiej Stankievič}
\title{Домашнее задание}

\begin{document}
\hypersetup{
pdftitle = {Задание 10},
pdfauthor = {Alaksiej Stankievič},
pdfsubject = {домашнее задание}
}% End of hypersetup


\section{Формулы}
\begin{equation}
 e^x = 1 + \frac{x}{1!}+\frac{x^2}{2!}+\frac{x^3}{3!}+\frac{x^4}{4!}+\frac{x^5}{5!}+...+\frac{x^n}{n!}...
\end{equation}
\begin{equation}
 \sin{x} = \frac{x}{1!}-\frac{x^3}{3!}+\frac{x^5}{5!}-\frac{x^7}{7!}+\frac{x^9}{9!}+...+\frac{(-1)^{n}x^{2n+1}}{(2n+1)!}...
\end{equation}
\begin{equation}
 \cos{x} = 1 - \frac{x^2}{2!}+\frac{x^4}{4!}-\frac{x^6}{6!}+\frac{x^8}{8!}+...+\frac{(-1)^{n}x^{2n}}{(2n)!}...
\end{equation}

Ряд сходится с заданной точностью, когда новый прибавляемый член становится по модулю меньше точности.

\section{Задача}
Посчитать соответствующую подзаданию функцию и сравнить с функцией, реализованной в стандартной библиотеке \verb|math.h|. Результаты представить в виде таблицы. В первом столбце $x$, во втором рассчитанное вами значение, в третьем значение библиотечной функции, а в четвёртом разность между ваши и библиотечным значением. Точность $\varepsilon$ (в программировании \verb|eps|) можете указать как в тексте программы, так и предложить ввести пользователю.

\begin{enumerate}
 \item Рассчитайте значение $e^x$\footnote{В стандартной библиотеке эта функция называется $\exp$. $\exp(x) \equiv e^x$}.
 \item Рассчитайте значение $\sin(x)$
 \item Рассчитайте значение $\cos(x)$
\end{enumerate}



\end{document}
