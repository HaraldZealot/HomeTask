\documentclass[12pt]{article} 
\usepackage[xetex, a4paper, left=2cm, right=2cm, top=2cm,bottom=2cm]{geometry}
\usepackage[cm-default]{fontspec}
\usepackage{xunicode}

%\tolerance=1000
%\emergencystretch=0.74cm 

\usepackage{polyglossia}
\setdefaultlanguage[spelling=modern]{russian}
\setotherlanguage{english} 
\defaultfontfeatures{Scale=MatchLowercase,Ligatures=TeX}  %% устанавливает поведение шрифтов по умолчанию  
\newfontfamily\cyrillicfont{Linux Libertine} 
\setromanfont[Mapping=tex-text]{Linux Libertine}
\setsansfont[Mapping=tex-text]{Linux Biolinum}
\setmonofont{DejaVu Sans Mono}
%\newfontfamily\cyrillicfont{Liberation Mono} 

%\usepackage{makecell}

%\usepackage{titlesec}
%\newcommand{\sectionbreak}{\clearpage}

%\renewcommand{\thesection}{\Alph{section}}
%\newcount\wd    \wd=\textwidth \multiply\wd by 8 \divide\wd by 17

\usepackage{minted}
\usemintedstyle{friendly}

\usepackage[unicode, pdfborder={0 0 0 0}]{hyperref}

\author{Alaksiej Stankievič}
\title{Домашнее задание}

\begin{document}
\hypersetup{
pdftitle = {Задание 15b},
pdfauthor = {Alaksiej Stankievič},
pdfsubject = {домашнее задание}
}% End of hypersetup


\section{Задачи}

Во всех задачах использовать прием объявления массива большого размера с реальным использованием лишь части. Размер
части спрашивается у пользователя. Также у пользователя есть выбор вводить массив вручную или генерировать. Напоминаю,
что всё должно быть разбито на функции.

\begin{enumerate}
  \item В массиве первый положительный элемент и последний отрицательный элемент переставить местами.
  \item В массиве найти максимальный среди четных элементов и минимальный среди нечетных.
  \item В массиве максимальные элементы, являющиеся четными числами, заменить значениями их индексов.
  \item В массиве все четные элементы заменить максимальным элементом,  а нечетные --- минимальным.
  \item В массиве найти последний четный положительный элемент, кратный заданному числу $p$, заменить его
индексом и поставить в  конец массива.
  \item Из всех участков массива , сплошь состоящих из нулей, выбрать самый длинный и отметить индексы его начала и
конца.
  \item В массиве найти максимальный из элементов, встречающихся только один раз.
  \item В массиве найти минимальное из чисел, встречающихся более одного раза.
  \item В массиве подсчитать количество элементов, встречающихся более одного раза.
  \item В массиве определить максимальную длину последовательности равных элементов.
  \item Дан массив $a$. Получить массив $b$, $i$-элемент которого равен среднему арифметическому первых $i$ элементов
массива  $a$.
\end{enumerate}



\end{document}
