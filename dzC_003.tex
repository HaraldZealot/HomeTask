\documentclass[12pt]{article} 
\usepackage[xetex, a4paper, left=2cm, right=2cm, top=2cm,bottom=2cm]{geometry}
\usepackage[cm-default]{fontspec}
\usepackage{xunicode}

%\tolerance=1000
%\emergencystretch=0.74cm 

\usepackage{polyglossia}
\setdefaultlanguage[spelling=modern]{russian}
\setotherlanguage{english} 
\defaultfontfeatures{Scale=MatchLowercase,Ligatures=TeX}  %% устанавливает поведение шрифтов по умолчанию  
\newfontfamily\cyrillicfont{Linux Libertine} 
\setromanfont[Mapping=tex-text]{Linux Libertine}
\setsansfont[Mapping=tex-text]{Linux Biolinum}
\setmonofont{DejaVu Sans Mono}
%\newfontfamily\cyrillicfont{Liberation Mono} 

%\usepackage{makecell}

%\usepackage{titlesec}
%\newcommand{\sectionbreak}{\clearpage}

%\renewcommand{\thesection}{\Alph{section}}
%\newcount\wd    \wd=\textwidth \multiply\wd by 8 \divide\wd by 17

\usepackage{minted}
\usemintedstyle{friendly}
\renewcommand\listingscaption{Код}
\newminted{bash}{frame=lines}
\newminted{c}{frame=leftline}

\author{Alaksiej Stankievič}
\title{Домашние задание 3}

\begin{document}
 \section{Запрограммировать на языке C}
 \begin{enumerate}
 \item Вывести на экран стихотворение Чарльза Буковского так, как показано в листинге \ref{lst:poetry}
 \begin{listing}[H]
  \begin{center}
   \begin{minted}[frame=lines]{text}
       And The Moon And The Stars And The World
                 by Charles Bukowski
                 
  Long walks at night--
  that's what good for the soul:
  peeking into windows
  watching tired housewives
  trying to fight off
  their beer-maddened husbands.
   \end{minted}
  \end{center}
  \caption{Стихотворение.}
  \label{lst:poetry}
 \end{listing}

  \item Написать программу, которая предлагает пользователю ввести два числа, а затем выводит 5 арифметических операций 
в следующем виде:
\begin{listing}[H]
 \begin{center}
  \begin{bashcode}
7 + 5 = 12
7 - 5 = 2
7 * 5 = 35
7 / 5 = 1
7 % 5 = 2  
  \end{bashcode}
 \end{center}
 \caption{Вывод программы, если пользователь ввёл 7 и 5.}
 \label{lst:sampleoutput}
\end{listing}

  \item Пользователь вводит радиус основания и высоту (можно целыми числами). Вывести объём и полную поверхность конуса 
(дробными числами с форматированием).
  \item Пользователь вводит 4 числа, а вы должны вывести числа большие в 2 раза по диагонали. Например пользователь ввёл
  2 7 0 3, тогда на экране будет 
\begin{listing}[H]
  \begin{center}
   \begin{minted}[frame=lines]{text}
4
 14
  0
   9
   \end{minted}
  \end{center}
  \caption{Диагональный вывод.}
  \label{lst:diagonal}
 \end{listing}
  
\end{enumerate}


\end{document}
