\documentclass[12pt]{article} 
\usepackage[xetex, a4paper, left=2cm, right=2cm, top=2cm,bottom=2cm]{geometry}
\usepackage[cm-default]{fontspec}
\usepackage{xunicode}

%\tolerance=1000
%\emergencystretch=0.74cm 

\usepackage{polyglossia}
\setdefaultlanguage[spelling=modern]{russian}
\setotherlanguage{english} 
\defaultfontfeatures{Scale=MatchLowercase,Ligatures=TeX}  %% устанавливает поведение шрифтов по умолчанию  
\newfontfamily\cyrillicfont{Linux Libertine} 
\setromanfont[Mapping=tex-text]{Linux Libertine}
\setsansfont[Mapping=tex-text]{Linux Biolinum}
\setmonofont{DejaVu Sans Mono}
%\newfontfamily\cyrillicfont{Liberation Mono} 

%\usepackage{makecell}

%\usepackage{titlesec}
%\newcommand{\sectionbreak}{\clearpage}

%\renewcommand{\thesection}{\Alph{section}}
%\newcount\wd    \wd=\textwidth \multiply\wd by 8 \divide\wd by 17

\usepackage{minted}
\usemintedstyle{friendly}

\author{Alaksiej Stankievič}
\title{Лабораторная работа на полиморфизм}

\begin{document}
 Запрограммировать на языке С. \textbf{Массивы не использовать}.
 \begin{enumerate}
  \item Вывести в виде таблицы (например по 16 символов в строке) первую половину таблицы ASCII (символы с кодами от 0 до 127 включительно).
  \item Пользователь вводит число $x$ и степень $p$. Вывести $x^p$. Степень целая (т.е. может быть отрицательной).
  \item Пользователь вводит число $n\geq1$. Затем вводит $n$ чисел. Вывести максимальное и минимальное из этих чисел.
  \item Пользователь вводит число $n\geq0$. Программа выводит $n$ чисел Фибоначчи\footnote{Напоминаю, числа Фибоначчи 
это последовательность 1, 1, 2, 3, 5, 8, 13, 21, 34, 55... Первых два числа равны единицам, каждое последующее сумма 
двух предыдущих}.
  \item Написать программу угадайку. Компьютер загадывает произвольное число от 1 до 99 включительно, а пользователь 
пытается его угадать. Компьютер даёт подсказки (<<перебор>>, <<недобор>>) и считает число попыток угадывания. В 
результате выдаётся число попыток за которые пользователь смог угадать число, и предлагается сыграть повторно.
 \end{enumerate}

\end{document}
