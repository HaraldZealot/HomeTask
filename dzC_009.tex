\documentclass[12pt]{article} 
\usepackage[xetex, a4paper, left=2cm, right=2cm, top=2cm,bottom=2cm]{geometry}
\usepackage[cm-default]{fontspec}
\usepackage{xunicode}

%\tolerance=1000
%\emergencystretch=0.74cm 

\usepackage{polyglossia}
\setdefaultlanguage[spelling=modern]{russian}
\setotherlanguage{english} 
\defaultfontfeatures{Scale=MatchLowercase,Ligatures=TeX}  %% устанавливает поведение шрифтов по умолчанию  
\newfontfamily\cyrillicfont{Linux Libertine} 
\setromanfont[Mapping=tex-text]{Linux Libertine}
\setsansfont[Mapping=tex-text]{Linux Biolinum}
\setmonofont{Ubuntu Mono}
%\newfontfamily\cyrillicfont{Liberation Mono} 

%\usepackage{makecell}

%\usepackage{titlesec}
%\newcommand{\sectionbreak}{\clearpage}

%\renewcommand{\thesection}{\Alph{section}}
%\newcount\wd    \wd=\textwidth \multiply\wd by 8 \divide\wd by 17

\usepackage{minted}
\usemintedstyle{friendly}
\renewcommand\listingscaption{Консольный вывод}


\usepackage[unicode, pdfborder={0 0 0 0}]{hyperref}

\author{Alaksiej Stankievič}
\title{Домашнее задание}

\begin{document}
\hypersetup{
pdftitle = {Задание 9},
pdfauthor = {Alaksiej Stankievič},
pdfsubject = {домашнее задание}
}% End of hypersetup

 Запрограммировать на языке С. Вызов функции system("cls") на windows или system("clear") на  linux очищает всю консоль.
 \section{более простые задания}
 \begin{enumerate}
\item Пользователь вводит два натуральных числа $m, n$. Заполнить поле $m\times{}n$ звёздочками в шахматном порядке.
  \item Саша и Вася положили в банк одинаковую сумму: 1000\$. Саша положил под простой\footnote{Процент всегда 
  начисляется от изначальной суммы на вкладе.} процент, ставка равна 6\%, а Вася под сложный\footnote{Процент 
начисляется от текущей суммы на вкладе.}, ставка --- 4\%. Процент начисляется ежеквартально (каждые три месяца). 
Найдите 
через какое время Вася будет иметь на счету большую сумму. Выведете  историю изменения суммы на счёте, до момента, 
когда 
Вася станет богаче, и ещё на протяжении трёх кварталов. Счёт выводить с точностью до центов, но сами деньги хранить 
внутри программы как целые числа.
  \item Проверить является ли введённое пользователем число палиндромом\footnote{Палиндром это число или фраза, 
  которые одинаково читаются в обоих направлениях. Например, число 121 палиндром, а 112 --- нет.}.
  \item Иван Васильевич зашёл в комнату и обнаружил, что туда залетели мухи. Он начал выгонять мух размахивая 
  полотенцем. Через некоторое время он устал и процес изгнания замедлился на 20\%. Вывести через какое время Иван 
Васильевич выгонит всех мух либо сообщить о том, что это не возможно. Пользователь вводит $N$ изначальное число мух, 
$exorcismSpeed$ изначальное количество выгоняемых полотенцем мух в минуту, $fatigueTime$ время через которое Иван 
Васильевич устаёт, $flyReturnSpeed$ число мух в минуту которые залетают назад.
  \item Составьте небольшой текстовый квест, или интерактивную книгу. Текст лучше делать транслитом или по английски. 
  В качестве примера смотрите \ref{lst:firstSC}, \ref{lst:secondSC} и \ref{lst:thirdSC}.
    \begin{listing}[H]
\begin{minted}[frame=single]{text}
Вы Илья Муромец, вам надоело лежать на печи и вы отправились за подвигами
\end{minted}
\caption{Первый экран}
\label{lst:firstSC}
\end{listing}
\begin{listing}[H]
\begin{minted}[frame=single]{text}
Подъехали вы к камню у развилки и прочли:
Налево поедешь себя потеряешь, направо поедешь коня потеряешь,
прямо поедешь неживой будешь.

1. налево\\
2. прямо\\
3. направо\\
Ваш выбор?
\end{minted}
\caption{Второй экран}
\label{lst:secondSC}
\end{listing}
\begin{listing}[H]
\begin{minted}[frame=single]{text}
Проехали вы пятирик шагов да угодили в волчью яму. А вас ведь предупреждали :)))
\end{minted}
\caption{Третий экран, пользователь ввёл 2.}
\label{lst:thirdSC}
\end{listing}
  \item Реализовать простой калькулятор. Пользователь вводит число, затем операцию в виде символа ('+', '-', '*', '/'), после второе число. Вывести результат операции.
 \end{enumerate}
 
 \section{более сложные задания}
 \begin{enumerate}
  \item Нарисовать ёлочку. Число ярусов вводит пользователь. Смотрите рисунок \ref{lst:tannen}.
\begin{listing}[H]
\begin{center}
\begin{minted}[frame=single]{text}
    *
   ***
    *
   * *
  *****
    *
   * *
  *   *
 *******
    *
   * *
  *   *
 *     *
*********
   * *
   * *
\end{minted}
\end{center}

\caption{Четырёхярусная ёлочка.}
\label{lst:tannen}
\end{listing}

  \item Реализовать более сложный калькулятор. Пользователь может вводить произвольное число чисел и операций 
  (поочерёдно), после введения символа '=' выдаётся конечный результат, и калькулятор сбрасывается в начало. Если 
вместо операции ввести символ 'c', то калькулятор сбрасывается в начало, при вводе символа 'e' вместо операции 
калькулятор завершает работу.
 \item Перевести введённое пользователем число в 12-ичную систему счисления.
 \item Придумать и запрограммировать задачу <<для души>>.
 \end{enumerate}


\end{document}
