\documentclass[12pt]{article} 
\usepackage[xetex, a4paper, left=2cm, right=2cm, top=2cm,bottom=2cm]{geometry}
\usepackage[cm-default]{fontspec}
\usepackage{xunicode}

%\tolerance=1000
%\emergencystretch=0.74cm 

\usepackage{polyglossia}
\setdefaultlanguage[spelling=modern]{russian}
\setotherlanguage{english} 
\defaultfontfeatures{Scale=MatchLowercase,Ligatures=TeX}  %% устанавливает поведение шрифтов по умолчанию  
\newfontfamily\cyrillicfont{Linux Libertine} 
\setromanfont[Mapping=tex-text]{Linux Libertine}
\setsansfont[Mapping=tex-text]{Linux Biolinum}
\setmonofont{DejaVu Sans Mono}
%\newfontfamily\cyrillicfont{Liberation Mono} 

%\usepackage{makecell}

%\usepackage{titlesec}
%\newcommand{\sectionbreak}{\clearpage}

%\renewcommand{\thesection}{\Alph{section}}
%\newcount\wd    \wd=\textwidth \multiply\wd by 8 \divide\wd by 17

\usepackage{minted}
\usemintedstyle{friendly}
\renewcommand\listingscaption{Код}
\newminted{bash}{frame=lines}
\newminted{c}{frame=leftline}

\usepackage[unicode, pdfborder={0 0 0 0}]{hyperref}

\author{Alaksiej Stankievič}
\title{Домашнее задание}

\begin{document}
\hypersetup{
pdftitle = {Задание 24},
pdfauthor = {Alaksiej Stankievič},
pdfsubject = {домашнее задание}
}% End of hypersetup

\section{Задание}


\begin{enumerate}
 \item Реализовать калькулятор коммандной строки для произвольных инфиксных 
(наших обычных) выражений. Должны быть поддержаны операции сложения, вычитания, 
умножения и возведения в степень, а также круглые скобки. Для работы может 
оказаться полезным использование тэгированных структур и функция преобразования 
строки в вещественное число из стандартной библиотеки.
 \item Реализовать два алгоритма планировщика процессов операционной системы. 
На входе есть текстовый файл в первой строке которого указаном число 
моделируемых процессов, а в последуещих строках для процессов заданы 
идентификатор (целое число), время прихода на обслуживание и число квантов 
времени необходимое процессу для выполнения задачи (тоже целые числа).
 \begin{enumerate}
  \item Алгоритм FCFS, первым пришёл первым обслужен.
  \item Алгоритм round robin (карусельный), для обслуживания одного процесса 
выделяется не более заданного числа квантов времени.
 \end{enumerate}
 \item Реализовать программу обхода графа. Граф задан в текстовом файле в 
первой строке которого указано число верши и число рёбер, а в последующих 
строках описано каждое ребро в виде пары двух чисел --- вершин графа связанных 
данным ребром, вершины нумеруются с единицы. Вывести в строку через пробел 
вершины в порядке соответсвующего обхода начиная с первой.
\begin{enumerate}
 \item Реализовать обход в глубину.
 \item Реализовать обход в ширину.
\end{enumerate}
 \item  Сгенерировать случайный лабиринт в котором отсутствуют замкнутые 
пространства (области из которых недостижим выход).
\end{enumerate}




\end{document}
