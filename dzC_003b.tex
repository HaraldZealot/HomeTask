\documentclass[12pt]{article} 
\usepackage[xetex, a4paper, left=2cm, right=2cm, top=2cm,bottom=2cm]{geometry}
\usepackage[cm-default]{fontspec}
\usepackage{xunicode}

%\tolerance=1000
%\emergencystretch=0.74cm 

\usepackage{polyglossia}
\setdefaultlanguage[spelling=modern]{russian}
\setotherlanguage{english} 
\defaultfontfeatures{Scale=MatchLowercase,Ligatures=TeX}  %% устанавливает поведение шрифтов по умолчанию  
\newfontfamily\cyrillicfont{Linux Libertine} 
\setromanfont[Mapping=tex-text]{Linux Libertine}
\setsansfont[Mapping=tex-text]{Linux Biolinum}
\setmonofont{DejaVu Sans Mono}
%\newfontfamily\cyrillicfont{Liberation Mono} 

%\usepackage{makecell}

%\usepackage{titlesec}
%\newcommand{\sectionbreak}{\clearpage}

%\renewcommand{\thesection}{\Alph{section}}
%\newcount\wd    \wd=\textwidth \multiply\wd by 8 \divide\wd by 17

\usepackage{minted}
\usemintedstyle{friendly}
\renewcommand\listingscaption{Код}
\newminted{bash}{frame=lines}
\newminted{c}{frame=leftline}

\author{Alaksiej Stankievič}
\title{Домашние задание 3b}

\begin{document}
 \section*{Строковые переменные}
 Для выполнения второго задания вам понадобятся строковые переменные, тема которая ещё будет 
нескоро\footnote{Предупреждаю, работа со строками отличается от работы с остальным видом переменных, в своё время мы 
это подробно изучим}, поэтому показываю простой способ объявить строковые переменные:
\begin{listing}[H]
 \begin{center}
  \begin{ccode}
    char a[]="Hello", b[]="world!";
  \end{ccode}
 \end{center}
 \caption{Строковые переменные}
 \label{lst:stringvariables}
\end{listing}
Важно!: наличие квадратных скобок и инициализация\footnote{первое присвоение переменной} прямо в строке с объявлением, 
по другому это не будет работать.

 \section*{Запрограммировать на языке C}
 \begin{enumerate}
 \item Вывести на экран три различных момента времени (по вашему желанию) в формате чч:мм:сс
 \begin{listing}[H]
  \begin{minted}[frame=lines]{text}
15:07:03
03:47:08
23:11:00
  \end{minted}
\caption{Пример вывода времени}
\label{lst:threetimes}
 \end{listing}
 \item  Используя 3 строковые переменных для каждого слова, вывести все возможные фразы: мама мыла раму, мыла раму мама 
и т.д.
 \item Вывести таблицу как минимум из четырёх человек, с указанием изменения (может быть как прибавка так уменьшение) 
веса в килограммах с точностью до грамма (три знака после запятой).
 \begin{listing}[H]
  \begin{minted}[frame=lines]{text}
   Jack |  2.148
Wiiliam | -1.769
    Sue |  3.015
 Judith | -5.200
  \end{minted}
\caption{Пример таблицы}
\label{lst:weighttable}
 \end{listing}
 \item (бонусное) сделать задачу про выведение чисел по диагонали из прошлой домашней использую * в спецификаторе 
вывода.
\end{enumerate}



\end{document}
