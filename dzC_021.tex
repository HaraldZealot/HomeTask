\documentclass[12pt]{article} 
\usepackage[xetex, a4paper, left=2cm, right=2cm, top=2cm,bottom=2cm]{geometry}
\usepackage[cm-default]{fontspec}
\usepackage{xunicode}

%\tolerance=1000
%\emergencystretch=0.74cm 

\usepackage{polyglossia}
\setdefaultlanguage[spelling=modern]{russian}
\setotherlanguage{english} 
\defaultfontfeatures{Scale=MatchLowercase,Ligatures=TeX}  %% устанавливает поведение шрифтов по умолчанию  
\newfontfamily\cyrillicfont{Linux Libertine} 
\setromanfont[Mapping=tex-text]{Linux Libertine}
\setsansfont[Mapping=tex-text]{Linux Biolinum}
\setmonofont{DejaVu Sans Mono}
%\newfontfamily\cyrillicfont{Liberation Mono} 

%\usepackage{makecell}

%\usepackage{titlesec}
%\newcommand{\sectionbreak}{\clearpage}

%\renewcommand{\thesection}{\Alph{section}}
%\newcount\wd    \wd=\textwidth \multiply\wd by 8 \divide\wd by 17

\usepackage{minted}
\usemintedstyle{friendly}
\renewcommand\listingscaption{Код}
\newminted{bash}{frame=lines}
\newminted{c}{frame=leftline}

\usepackage[unicode, pdfborder={0 0 0 0}]{hyperref}

\author{Alaksiej Stankievič}
\title{Домашнее задание}

\begin{document}
\hypersetup{
pdftitle = {Задание 21},
pdfauthor = {Alaksiej Stankievič},
pdfsubject = {домашнее задание}
}% End of hypersetup

\section{Задание}


\begin{enumerate}
 \item Реализовать собственную функцию strlen.
 \item Реализовать собственную функцию strcmp.
 \item Реализовать собственную функцию strcpy.
 \item Напечатать без повторения слова текста, у которых первая и последняя буквы совпадают.
 \item Определить, каких букв, гласных или согласных, больше в каждом предложении текста.
 \item В тексте найти слова, длина которых  простое число, и заменить их  ***…* .  Количество  *  должно соответствовать
длине слова
 \item Реализовать упрощённую версию утилиты cat. Она должна выполнять основной функционал прототипа (склеивать файлы и
 выводить их на стандартный вывод), а также поддерживать опции \verb|-n|, \verb|--number| и \verb|--help|.
 \end{enumerate}



\end{document}
