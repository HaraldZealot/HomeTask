\documentclass[12pt]{article} 
\usepackage[xetex, a4paper, left=2cm, right=2cm, top=2cm,bottom=2cm]{geometry}
\usepackage[cm-default]{fontspec}
\usepackage{xunicode}

%\tolerance=1000
%\emergencystretch=0.74cm 

\usepackage{polyglossia}
\setdefaultlanguage[spelling=modern]{russian}
\setotherlanguage{english} 
\defaultfontfeatures{Scale=MatchLowercase,Ligatures=TeX}  %% устанавливает поведение шрифтов по умолчанию  
\newfontfamily\cyrillicfont{Linux Libertine} 
\setromanfont[Mapping=tex-text]{Linux Libertine}
\setsansfont[Mapping=tex-text]{Linux Biolinum}
\setmonofont{DejaVu Sans Mono}
%\newfontfamily\cyrillicfont{Liberation Mono} 

%\usepackage{makecell}

%\usepackage{titlesec}
%\newcommand{\sectionbreak}{\clearpage}

%\renewcommand{\thesection}{\Alph{section}}
%\newcount\wd    \wd=\textwidth \multiply\wd by 8 \divide\wd by 17

\usepackage{minted}
\usemintedstyle{friendly}

\usepackage[unicode, pdfborder={0 0 0 0}]{hyperref}

\author{Alaksiej Stankievič}
\title{Домашнее задание}

\begin{document}
\hypersetup{
pdftitle = {Задание 14b},
pdfauthor = {Alaksiej Stankievič},
pdfsubject = {домашнее задание}
}% End of hypersetup

Во всех задачах, если специально не оговорено иное, необходимо разбить программу на несколько файлов.

\section{Задачи}

\begin{enumerate}
 \item Создайте проект с использованием make. В отдельный файл вынесите следующие функции:
 \begin{enumerate}
  \item подсчёт числа цифр в числе (для 123 это 3, для 0 это 1)
  \item отделение первой цифры (самой левой, для 123 это 1)
  \item отделение всех цифр кроме первой (для 123 это 23)
  \item приписывания к числу цифры в конце (для 23 и 4 это 234)
  \item циклического сдвига цифр в числе (для 123 это 231)
 \end{enumerate}
  Продемонстрируйте в функции main работу всех этих функций.
 \item Использую модуль совместимости реализуйте анимацию расширяющегося и сужающегося в центре квадратика.
 \item Вы хотите взломать сейф особой конструкции, ключевая комбинация в нём задаётся перестановкой цифр в заданном 
числе. Напишите программу помощник, которая по введённому числу распечатает все перестановки цифр в 
числе\footnote{Указание: воспользуйтесь рекурсией и функциями которые вы написали в первом задании.}.
 \item(бонусная) Пользователь вводит число, это ширина в символах спирали. Высота всегда на единицу больше. Каждый 
виток спирали печатается следующей буквой, первая буква А. Смотрите пример \ref{lst:spiral}. Распечатайте спираль 
(массивы и матрицы не использовать).
\begin{listing}[H]
\begin{center}
\begin{minted}[frame=single]{text}
AAAAAAAAAAAAAAAA
               A
ABBBBBBBBBBBBB A
A            B A
A BCCCCCCCCC B A
A B        C B A
A B CDDDDD C B A
A B C    D C B A
A B C DE D C B A
A B C D  D C B A
A B C DDDD C B A
A B C      C B A
A B CCCCCCCC B A
A B          B A
A BBBBBBBBBBBB A
A              A
AAAAAAAAAAAAAAAA
\end{minted}
\end{center}
\caption{Спираль $16\times{}17$}
\label{lst:spiral}
\end{listing}
\end{enumerate}

\end{document}
