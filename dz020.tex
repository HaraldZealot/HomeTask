\documentclass[12pt]{article} 
\usepackage[xetex, a4paper, left=2cm, right=2cm, top=2cm,bottom=2cm]{geometry}
\usepackage[cm-default]{fontspec}
\usepackage{xunicode}

%\tolerance=1000
%\emergencystretch=0.74cm 

\usepackage{polyglossia}
\setdefaultlanguage[spelling=modern]{russian}
\setotherlanguage{english} 
\defaultfontfeatures{Scale=MatchLowercase,Ligatures=TeX}  %% устанавливает поведение шрифтов по умолчанию  
\newfontfamily\cyrillicfont{Linux Libertine} 
\setromanfont[Mapping=tex-text]{Linux Libertine}
\setsansfont[Mapping=tex-text]{Linux Biolinum}
\setmonofont{DejaVu Sans Mono}
%\newfontfamily\cyrillicfont{Liberation Mono} 

%\usepackage{makecell}

%\usepackage{titlesec}
%\newcommand{\sectionbreak}{\clearpage}

%\renewcommand{\thesection}{\Alph{section}}
%\newcount\wd    \wd=\textwidth \multiply\wd by 8 \divide\wd by 17

\usepackage{minted}
\usemintedstyle{friendly}
\renewcommand\listingscaption{Код}
\newminted{bash}{frame=lines}
\newminted{c}{frame=leftline}

\usepackage[unicode, pdfborder={0 0 0 0}]{hyperref}

\author{Alaksiej Stankievič}
\title{Домашнее задание}

\begin{document}
\hypersetup{
pdftitle = {Задание 20},
pdfauthor = {Alaksiej Stankievič},
pdfsubject = {домашнее задание}
}% End of hypersetup

\section{Задание}


\begin{enumerate}
 \item Найти в случайной матрице седловую точку если она там есть. Седловой точкой называется элемент матрицы который является минимумом в строке и максимумом в столбце. Матрица должна быть динамической, размеры задаёт пользователь, числа случайные. Реализовать две версии программы на каждый из способов реализации динамической матрицы\footnote{Это в качестве упражнения, в дальнейшем пользуйтесь тем способом, который вам будет удобнее}.
 \item Реализовать работу со стеком с помощью функций, в листинге \ref{lst:pushfunction} приведен пример функции push, реализуйте также pop, onTop, isEmpty.
 \item Реализуйте поиск корня уравнения метод деления пополам (с возможностью прямого попадания в корень). Решите им три различных уравнения в одной программе.
 \item С помощью стандартной функции qsort, отсортируйте массив по убыванию, по возрастанию суммы цифр, по возрастанию количества делителей числа.
\end{enumerate}
\begin{listing}[H]
\begin{center}
\begin{ccode}
void push(int *stack, int size, int *top, int datum)
{
    assert(*top < size - 1);
    stack[++(*top)] = datum;
}
\end{ccode}
\end{center}
\caption{Функция добавления элемента в стек}
\label{lst:pushfunction}
\end{listing}


\end{document}
