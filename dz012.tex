\documentclass[12pt]{article} 
\usepackage[xetex, a4paper, left=2cm, right=2cm, top=2cm,bottom=2cm]{geometry}
\usepackage[cm-default]{fontspec}
\usepackage{xunicode}

%\tolerance=1000
%\emergencystretch=0.74cm 

\usepackage{polyglossia}
\setdefaultlanguage[spelling=modern]{russian}
\setotherlanguage{english} 
\defaultfontfeatures{Scale=MatchLowercase,Ligatures=TeX}  %% устанавливает поведение шрифтов по умолчанию  
\newfontfamily\cyrillicfont{Linux Libertine} 
\setromanfont[Mapping=tex-text]{Linux Libertine}
\setsansfont[Mapping=tex-text]{Linux Biolinum}
\setmonofont{DejaVu Sans Mono}
%\newfontfamily\cyrillicfont{Liberation Mono} 

%\usepackage{makecell}

%\usepackage{titlesec}
%\newcommand{\sectionbreak}{\clearpage}

%\renewcommand{\thesection}{\Alph{section}}
%\newcount\wd    \wd=\textwidth \multiply\wd by 8 \divide\wd by 17

\usepackage{minted}
\usemintedstyle{friendly}

\usepackage[unicode, pdfborder={0 0 0 0}]{hyperref}

\author{Alaksiej Stankievič}
\title{Домашнее задание}

\begin{document}
\hypersetup{
pdftitle = {Задание 12},
pdfauthor = {Alaksiej Stankievič},
pdfsubject = {домашнее задание}
}% End of hypersetup

\section{Необходимые сведения}
\subsection{Метод деления пополам}
Пускай надо найти корень какого-либо уравнения $f(x)=0$ на отрезке $[a, b]$, при этом корень на этом отрезке единственный\footnote{Также необходимо, чтобы функция была непрерывной.}. Тогда на концах отрезках функция обязательно принимает разные знаки. Если мы возмём точку $m$ по середине отрезка, то функция там принимает какой-то знак\footnote{Возможно оно там равна нулю, тогда мы попали в корень, но это реже бывает}. Значит корень заключён между серединой и тем из концов отрезка, который имеет другой знак. Процесс продолжаем до тех пор пока длина рассматриваемого отрезка не станет меньше заданной точности, тогда считаем, что корень приблизительно на середине отрезка.

Описание алгоритма в псведокоде:
\begin{itemize}
 \item Шаг 1. Выбрать точки $a$ и $b$. Посчитать в них функцию $f_a=f(a)$, $f_b=f(b)$
 \item Шаг 2. Найти середину отрезка $m=(a+b)/2$.
 \item Шаг 3. Посчитать функцию в середине $f_m=f(m)$
 \item Шаг 4. Если $f_m$ равна нулю, то перейти к шагу 9.
 \item Шаг 5. Если $f_a\cdot{}f_m<0$, то присвоить $f_b=f_m$ и $b=m$. 
 \item Шаг 6. Иначе $f_a=f_m$ и $a=m$.
 \item Шаг 7. Если $b-a>\varepsilon$ перейти к шагу 2.
 \item Шаг 8. Найти середину отрезка $m=(a+b)/2$.
 \item Шаг 9. $m$ --- искомый корень уравнения. Конец
\end{itemize}

\subsection{Квадратное уравнение}
Если задано квадратное уравнение:
\begin{equation}
 ax^2+bx+c=0,
\end{equation}
то можно вычислить его дискриминант $D$ по формуле:
\begin{equation}
 D=b^2-4ac.
\end{equation}
Если $D<0$, то корней нет, если $D=0$, то корень один $x=-b/2a$, и наконец если $D>0$, то два корня можн найти по следующим формулам:
\begin{equation}
 x_1=\frac{-b+\sqrt{D}}{2a},
\end{equation}
\begin{equation}
 x_1=\frac{-b-\sqrt{D}}{2a}.
\end{equation}




\section{Задачи от меня}

\begin{enumerate}
 \item Найдите квадратный корень методом деления пополам.
 \item Пользователь вводит коэффициенты $a$, $b$ и $c$. Решить квадратное уравнение $ax^2 + bx + c = 0$.
 \item Найти к чему стремится отношение двух соседних членов ряда Фибоначчи при увеличивающемся $n$.
\end{enumerate}

\section{Задачи от Ивана}

\subsection*{Лёгкий уровень}
\begin{enumerate}
 \item Вывести 10 раз слово \textit{Hello}.
 \item Имеется число 64, вывести пошаговое деление числа на 2 (т.е. вывод должен выглядеть как):
32
16
8
4
2
1
0.
\end{enumerate}

\subsection*{Средний уровень}
\begin{enumerate}
\item Пользователь вводит любое число, программа должна определить чётное число или нет. Если чётное --- вывести надпись \textit{Even}, если нечётное --- \textit{Odd}.
\item Пользователь вводит любое число. Отсчитать от этого числа до нуля. К примеру, пользователь вводит 5, программа в ответ выводит:
4
3
2
1
0.
\end{enumerate}

\subsection*{Сложный уровень}
\begin{enumerate}
 \item Пользователь вводит любое число и любой делитель (например 10 и 2), программа выводит результаты деления, а в конце, то есть когда число дальше нацело не делится, пишет \textit{Finish}.
 \item Влад и Иван бегут наперегонки, пользователь вводит скорость Влада и Вани в м/с, а также длительность забега в метрах. Влад устаёт через 57\% дистанции, Ваня через 63\%. После наступления усталости Ваня бежит на 30\% медленнее, Влад на 28\%. Определить, кто придёт к финишу первым.
\end{enumerate}



\end{document}
