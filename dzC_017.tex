\documentclass[12pt]{article} 
\usepackage[xetex, a4paper, left=2cm, right=2cm, top=2cm,bottom=2cm]{geometry}
\usepackage[cm-default]{fontspec}
\usepackage{xunicode}

%\tolerance=1000
%\emergencystretch=0.74cm 

\usepackage{polyglossia}
\setdefaultlanguage[spelling=modern]{russian}
\setotherlanguage{english} 
\defaultfontfeatures{Scale=MatchLowercase,Ligatures=TeX}  %% устанавливает поведение шрифтов по умолчанию  
\newfontfamily\cyrillicfont{Linux Libertine} 
\setromanfont[Mapping=tex-text]{Linux Libertine}
\setsansfont[Mapping=tex-text]{Linux Biolinum}
\setmonofont{DejaVu Sans Mono}
%\newfontfamily\cyrillicfont{Liberation Mono} 

%\usepackage{makecell}

%\usepackage{titlesec}
%\newcommand{\sectionbreak}{\clearpage}

%\renewcommand{\thesection}{\Alph{section}}
%\newcount\wd    \wd=\textwidth \multiply\wd by 8 \divide\wd by 17

\usepackage{minted}
\usemintedstyle{friendly}
\renewcommand\listingscaption{Код}
\newminted{bash}{frame=lines}
\newminted{c}{frame=leftline}

\usepackage{xcolor}

\usepackage[unicode]{hyperref}

\author{Alaksiej Stankievič}
\title{Домашнее задание}

\begin{document}
\hypersetup{
pdftitle = {Задание 17},
pdfauthor = {Alaksiej Stankievič},
pdfsubject = {домашнее задание},
hidelinks
}% End of hypersetup

\section{Стек}
Стек --- это абстрактная структура данных поддерживающая протокол LIFO (last input first output --- последним зашёл,
первым вышел). Абстрактность в определение означает, что нас не интересует внутренняя реализация, достаточно чтобы нам
был предоставлен интерфейс. Интерфейс стека содержит в обязательном порядке два действия:
\begin{enumerate}
 \item поместить данные в конец стека (push),
 \item извлечь данные из стека (pop).
\end{enumerate}
Часто эти действия реализуются в виде соответствующих функций, но мы ещё не знаем механизм передачи по указателю, чтобы
грамотно их записать, поэтому пока мы будем писать прямым кодом и копипастом. Также часто функциональность по работе со
стеком содержит действия по просмотру вершины без извлечения, проверки на пустоту и полной очистки стека.

Как уже было указано внутренняя реализация стека может быть различной. Одной из таких является реализация с помощью
массива, её мы рассмотрим подробнее.

Пусть у нас есть массив, который мы обозначим как stack, и переменная, которую назовём top (то есть верхушка или вершина
стека). Изначально top имеет значение -1, а массив stack хоть и содержит информацию, но считается пустым.
Тогда операция push выглядит как:
\begin{listing}[H]
\begin{center}
\begin{ccode}
++top;
stack[top] = input;
\end{ccode}
\end{center}
\caption{Развёрнутая версия push}
\label{lst:pushext}
\end{listing}
Что можно сократить до одной строки.
\begin{listing}[H]
\begin{center}
\begin{ccode}
stack[++top] = input;
\end{ccode}
\end{center}
\caption{Однострочная версия push}
\label{lst:pushsingle}
\end{listing}

Аналогично выглядит и операции pop в развёрнутом (код~\ref{lst:popext}) и однострочном виде (код~\ref{lst:popsingle}).
\begin{listing}[H]
\begin{center}
\begin{ccode}
output = stack[top];
--top;
\end{ccode}
\end{center}
\caption{Развёрнутая версия pop}
\label{lst:popext}
\end{listing}
\begin{listing}[H]
\begin{center}
\begin{ccode}
output = stack[top--];
\end{ccode}
\end{center}
\caption{Однострочная версия pop}
\label{lst:popsingle}
\end{listing}

\section{Обратная польская запись}

\href{
https://ru.wikipedia.org/wiki/%D0%9E%D0%B1%D1%80%D0%B0%D1%82%D0%BD%D0%B0%D1%8F_%D0%BF%D0%BE%D0%BB%D1%8C%D1%81%D0%BA%D0%B
0%D1%8F_%D0%B7%D0%B0%D0%BF%D0%B8%D1%81%D1%8C}{\emph{Обратная польская запись}}
--- это один из способов записи математических выражений. Удобен тем, что для него нет необходимости понятия приоритета
операций, так как само положение операторов целиком и полностью определяет порядок вычисления. С помощью стека обратная
польская запись легко вычисляется. Если встречается число, то оно заносится в стек, если встречается оператор, то
извлекается два элемента (учитывая порядок), над ними производится операция, результат которой заносится в стек. После
чтения всего выражения в стеке будет только одно число --- результат  выражения.

\section{Язык brainfuck}

Язык \href{https://ru.wikipedia.org/wiki/Brainfuck}{\emph{brainfuck}} --- это минималистичный эзотерический язык
программирования. Он работает в следующей модели: есть память в виде массива unsigned char некоторого размера,
изначально весь массив заполнен нулями. Есть пишуще-читающая головка, изначально установленная в нулевую позицию памяти,
ячейка на которую установлена головка является текущей.

В языке восемь операций

\begin{itemize}
 \item \verb|+| увеличивает значение текущей ячейки на 1 (с учётом переполнения).
 \item \verb|-| уменьшает значение текущей ячейки на 1 (с учётом переполнения).
 \item \verb|>| сдвигает головку на одну позицию вправо.
 \item \verb|<| сдвигает головку на одну позицию влево.
 \item \verb|.| выводит текущую ячейку на консоль как символ.
 \item \verb|,| вводит символ с клавиатуры в текущую ячейку.
 \item \verb|[| если текущая ячейка не ноль, то программа переходит на следующую позицию, если ноль, то переходит на
позицию следующую после соответствующей закрывающей скобочки.
 \item \verb|]| если текущая ячейка не ноль, то переходит на позицию следующую после соответствующей открывающей
 скобочки, если ноль, то программа переходит на следующую позицию.
\end{itemize}


\section{Задание}

\begin{enumerate}
 \item Пользователь вводит последовательность из открывающихся и закрывающихся скобочек 3 видов. Проверить
правильность растановки скобок.
 \item Пользователь вводит обратную польскую запись в виде одноциферных чисел и знаков без пробелов, в конце пишет
 знак <<=>>. Посчитать результат записи полпгая что числа вещественные.
 \item Реализовать интерпретатор упрощённой версии brainfuck (нет операций <<[>>, <<]>> и <<,>>, а операция <<;>>
 является признаком конца программы). Интерпретатор должен работать <<налету>>, то есть в цикле чтения программы он
должен её выполнять.
 \item Реализовать интерпретатор полной версии brainfuck. Признаком конца программы является символ перевода строки
 (используйте функции getchar() и putchar(), смотрите например
\href{http://www.cplusplus.com/reference/cstdio/}{\emph{здесь}}
).
\end{enumerate}

Для получения бонуса нужно в интерпретаторе brainfuck реализовать все возможные защиты.



\end{document}
