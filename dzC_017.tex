\documentclass[12pt]{article} 
\usepackage[xetex, a4paper, left=2cm, right=2cm, top=2cm,bottom=2cm]{geometry}
\usepackage[cm-default]{fontspec}
\usepackage{xunicode}

%\tolerance=1000
%\emergencystretch=0.74cm 

\usepackage{polyglossia}
\setdefaultlanguage[spelling=modern]{russian}
\setotherlanguage{english} 
\defaultfontfeatures{Scale=MatchLowercase,Ligatures=TeX}  %% устанавливает поведение шрифтов по умолчанию  
\newfontfamily\cyrillicfont{Linux Libertine} 
\setromanfont[Mapping=tex-text]{Linux Libertine}
\setsansfont[Mapping=tex-text]{Linux Biolinum}
\setmonofont{DejaVu Sans Mono}
%\newfontfamily\cyrillicfont{Liberation Mono} 

%\usepackage{makecell}

%\usepackage{titlesec}
%\newcommand{\sectionbreak}{\clearpage}

%\renewcommand{\thesection}{\Alph{section}}
%\newcount\wd    \wd=\textwidth \multiply\wd by 8 \divide\wd by 17

\usepackage{minted}
\usemintedstyle{friendly}

\usepackage[unicode, pdfborder={0 0 0 0}]{hyperref}

\author{Alaksiej Stankievič}
\title{Домашнее задание}

\begin{document}
\hypersetup{
pdftitle = {Задание 17},
pdfauthor = {Alaksiej Stankievič},
pdfsubject = {домашнее задание}
}% End of hypersetup

\section{Обратная польская запись}

\href{https://ru.wikipedia.org/wiki/%D0%9E%D0%B1%D1%80%D0%B0%D1%82%D0%BD%D0%B0%D1%8F_%D0%BF%D0%BE%D0%BB%D1%8C%D1%81%D0%BA%D0%B0%D1%8F_%D0%B7%D0%B0%D0%BF%D0%B8%D1%81%D1%8C}{Обратная польская запись} 
--- это один из способов записи математических выражений. Удобен тем, что для него нет необходимости понятия приоритета операций, так как само положение операторов целиком и полностью определяет порядок вычисления. С помощью стека обратная польская запись легко вычисляется. Если встречается число, то оно заносится в стек, если встречается оператор, то извлекается два элемента (учитывая порядок), над ними производится операция, результат которой заносится в стек. После чтения всего выражения в стеке будет только одно число --- результат  выражения.

\section{Язык brainfuck}

Язык \href{https://ru.wikipedia.org/wiki/Brainfuck}{brainfuck} --- это минималистичный эзотерический язык программирования. Он работает в следующей модели: есть память в виде массива unsigned char некоторого размера, изначально весь массив заполнен нулями. Есть пишуще-читающая головка, изначально установленная в нулевую позицию памяти, ячейка на которую установлена головка является текущей. 

В языке восемь операций

\begin{itemize}
 \item \verb|+| увеличивает значение текущей ячейки на 1 (с учётом переполнения).
 \item \verb|-| уменьшает значение текущей ячейки на 1 (с учётом переполнения).
 \item \verb|>| сдвигает головку на одну позицию вправо.
 \item \verb|<| сдвигает головку на одну позицию влево.
 \item \verb|.| выводит текущую ячейку на консоль как символ.
 \item \verb|,| вводит символ с клавиатуры в текущую ячейку.
 \item \verb|[| если текущая ячейка не ноль, то программа переходит на следующую позицию, если ноль, то переходит на позицию следующую после соответствующей закрывающей скобочки.
 \item \verb|]| если текущая ячейка не ноль, то переходит на позицию следующую после соответствующей открывающей скобочки, если ноль, то программа переходит на следующую позицию.
\end{itemize}


\section{Задание}

\begin{enumerate}
 \item Пользователь вводит обратную польскую запись в виде одноциферных чисел и знаков без пробелов, в конце пишет знак <<=>>. Посчитать результат записи полпгая что числа вещественные.
 \item Реализовать интерпретатор упрощённой версии brainfuck (нет операций <<[>>, <<]>> и <<,>>, а операция <<;>> является признаком конца программы). Интерпретатор должен работать <<налету>>, то есть в цикле чтения программы он должен её выполнять.
 \item Реализовать интерпретатор полной версии brainfuck. Признаком конца программы является символ перевода строки (используйте функции getchar() и putchar(), смотрите например \href{http://www.cplusplus.com/reference/cstdio/}{здесь} ).
\end{enumerate}

Для получения бонуса нужно в интерпретаторе brainfuck реализовать все возможные защиты.



\end{document}
