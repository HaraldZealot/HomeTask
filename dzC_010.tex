\documentclass[12pt]{article} 
\usepackage[xetex, a4paper, left=2cm, right=2cm, top=2cm,bottom=2cm]{geometry}
\usepackage[cm-default]{fontspec}
\usepackage{xunicode}

%\tolerance=1000
%\emergencystretch=0.74cm 

\usepackage{polyglossia}
\setdefaultlanguage[spelling=modern]{russian}
\setotherlanguage{english} 
\defaultfontfeatures{Scale=MatchLowercase,Ligatures=TeX}  %% устанавливает поведение шрифтов по умолчанию  
\newfontfamily\cyrillicfont{Linux Libertine} 
\setromanfont[Mapping=tex-text]{Linux Libertine}
\setsansfont[Mapping=tex-text]{Linux Biolinum}
\setmonofont{DejaVu Sans Mono}
%\newfontfamily\cyrillicfont{Liberation Mono} 

%\usepackage{makecell}

%\usepackage{titlesec}
%\newcommand{\sectionbreak}{\clearpage}

%\renewcommand{\thesection}{\Alph{section}}
%\newcount\wd    \wd=\textwidth \multiply\wd by 8 \divide\wd by 17

\usepackage{minted}
\usemintedstyle{friendly}

\usepackage[unicode, pdfborder={0 0 0 0}]{hyperref}

\author{Alaksiej Stankievič}
\title{Домашнее задание}

\begin{document}
\hypersetup{
pdftitle = {Задание 10},
pdfauthor = {Alaksiej Stankievič},
pdfsubject = {домашнее задание}
}% End of hypersetup


 Задачи от Левшинского Ивана, можно обращаться как ко мне, так и к нему с вопросами.
 \section*{Простые}
  \begin{enumerate}
   \item Вывести 10 раз любое число.
   \item Вывести сумму, вычитание, умножение двух любых чисел.
   \item Вывести число 10 в целом и дробном формате (то есть в виде 10 и 10.00000).
  \end{enumerate}
  
  \section*{Средние}
  \begin{enumerate}
   \item Пользователь вводит 2 числа, вывести сумму, вычитание, умножение и деление этих чисел. (Избежать деления на 0!).
   \item Используя цикл и switch  выполнить следующее:
    \begin{itemize}
     \item если пользователь вводит 1 вывести <<How are you, user?>>
     \item если пользователь вводит 2 вывести <<Bye-bye, user.>>
     \item если пользователь вводит 0 выйти из программы.
     \item если пользователь вводит что-либо другое, вывести <<What?>>
     Желательно печатать соответствующее меню.
    \end{itemize}
   \item Пользователь вводит два числа, определить максимальное и минимальное.
  \end{enumerate}
  
  \section*{Сложные}
  \begin{enumerate}
   \item Пользователь вводит по очереди 3 числа, вывести сумму, вычитание, умножение, деление и целочисленное деление этих чисел. (Избежать деления на 0!).
   \item Используя цикл и ветвление выполнить следующее: если сумма чисел, введённых пользователем, больше 100, вывести надпись <<Yoursum of numbers>100>>, если меньше: <<Your sum of numbers<100>>.
   \item У Захара и Вадима, по 10000\$ на счёте, оба положили деньги в банк под 13\%. Вывести количество денег Захара через 12 месяцев, количество денег Вадима через 7 месяцев и количество денег Влада, равное 63\% от суммы денег Захара и Вадима на 5 месяце.
  \end{enumerate}


\end{document}
