\documentclass[12pt]{article} 
\usepackage[xetex, a4paper, left=2cm, right=2cm, top=2cm,bottom=2cm]{geometry}
\usepackage[cm-default]{fontspec}
\usepackage{xunicode}

%\tolerance=1000
%\emergencystretch=0.74cm 

\usepackage{polyglossia}
\setdefaultlanguage[spelling=modern]{russian}
\setotherlanguage{english} 
\defaultfontfeatures{Scale=MatchLowercase,Ligatures=TeX}  %% устанавливает поведение шрифтов по умолчанию  
\newfontfamily\cyrillicfont{Linux Libertine} 
\setromanfont[Mapping=tex-text]{Linux Libertine}
\setsansfont[Mapping=tex-text]{Linux Biolinum}
\setmonofont{DejaVu Sans Mono}
%\newfontfamily\cyrillicfont{Liberation Mono} 

%\usepackage{makecell}

%\usepackage{titlesec}
%\newcommand{\sectionbreak}{\clearpage}

%\renewcommand{\thesection}{\Alph{section}}
%\newcount\wd    \wd=\textwidth \multiply\wd by 8 \divide\wd by 17

\usepackage{minted}
\usemintedstyle{friendly}
\renewcommand\listingscaption{Код}
\newminted{bash}{frame=lines}
\newminted{c}{frame=leftline}

\usepackage[unicode, pdfborder={0 0 0 0}]{hyperref}

\author{Alaksiej Stankievič}
\title{Домашнее задание}

\begin{document}
\hypersetup{
pdftitle = {Задание 27},
pdfauthor = {Alaksiej Stankievič},
pdfsubject = {домашнее задание}
}% End of hypersetup

\section{Игра <<жизнь>>}
Игра <<жизнь>> Коэна --- это поле клеточных автоматов которые эволиционируют 
некоторым образом. Клеточный автомат это конечный автомат (соответсвенно он 
имееет конечное число состояний) получающий в качестве входа состояния 
окружающих автоматов.

Коэн определил следующие правилы игры <<жизнь>>:
\begin{enumerate}
 \item Есть бесконечное квадртаное клеточное поле.
 \item Каждая клеточка может быть в двух состояниях: живая и мёртвая.
 \item У каждой клетки есть восемь соседей по кругу, их состояние влияет на 
состояние клетки.
 \item Если у пустой (мёртвой) клетки, имеется ровно 3 соседа, то она оживает, 
<<рождается>>.
 \item Если у живой клетки 2 или 3 соседа, то она продолжает жить, а в 
противном случае она умирает.
\end{enumerate}


\section{Задание}

Реализуйте программу с использованием библиотеки SDL2, моделирующую игру 
<<жизнь>>. Бесконечное поле замените полем с тороидальной 
топологией\footnote{Это когда при переходе через край мы оказываемся на 
противоположному краю}. Необходим режим расстановки, в котором пользователь 
может с помощью клавиш (и возможно мыши) сделать изначальную конфигурацию, и 
режим моделирования.





\end{document}
