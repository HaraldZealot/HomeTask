\documentclass[12pt]{article} 
\usepackage[xetex, a4paper, left=2cm, right=2cm, top=2cm,bottom=2cm]{geometry}
\usepackage[cm-default]{fontspec}
\usepackage{xunicode}

%\tolerance=1000
%\emergencystretch=0.74cm 

\usepackage{polyglossia}
\setdefaultlanguage[spelling=modern]{russian}
\setotherlanguage{english} 
\defaultfontfeatures{Scale=MatchLowercase,Ligatures=TeX}  %% устанавливает поведение шрифтов по умолчанию  
\newfontfamily\cyrillicfont{Linux Libertine} 
\setromanfont[Mapping=tex-text]{Linux Libertine}
\setsansfont[Mapping=tex-text]{Linux Biolinum}
\setmonofont{DejaVu Sans Mono}
%\newfontfamily\cyrillicfont{Liberation Mono} 

%\usepackage{makecell}

%\usepackage{titlesec}
%\newcommand{\sectionbreak}{\clearpage}

%\renewcommand{\thesection}{\Alph{section}}
%\newcount\wd    \wd=\textwidth \multiply\wd by 8 \divide\wd by 17

\usepackage{minted}
\usemintedstyle{friendly}

\usepackage[unicode, pdfborder={0 0 0 0}]{hyperref}

\author{Alaksiej Stankievič}
\title{Домашнее задание}

\begin{document}
\hypersetup{
pdftitle = {Задание 13},
pdfauthor = {Alaksiej Stankievič},
pdfsubject = {домашнее задание}
}% End of hypersetup

\section{Задачи}

\begin{enumerate}
 \item Написать программу с функциями вычисляющими НОД и НОК.
 \item Написать функцию возведение дробного числа в целую (в том числе отрицательную степень)\footnote{Хорошо, если реализуете двоичное возведение в степень}.
 \item Пользователь вводит $n$. В поле $n\times{}n$ нужно вывести диез если номер строки $i$ и номер столбца $j$ взаимно просты\footnote{Числа взаимно просты если их НОД равен 1.} и пробел в остальных случаях.
\end{enumerate}

\section{Задачи от Ивана}


\begin{enumerate}
 \item Вывести любые 10 букв одну под другой (через \verb|\n|, то есть):\\
A\\
G\\
M
 \item Пользователь вводит два числа, определить делится ли первое число на второе и делится ли второе на первое. Соответственно вывести a/b (если первое делится на второе), b/a (если второе делится на первое), a!/b (если первое не делится на второе), b!/a (если второе не делится на первое).
 \item Пользователь вводит 3 числа, $a$, $b$ и $c$, если $a + b>c$, вывести надпись ``a + b>c'', если меньше то ``a + b<c''.
\end{enumerate}

\end{document}
