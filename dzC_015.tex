\documentclass[12pt]{article} 
\usepackage[xetex, a4paper, left=2cm, right=2cm, top=2cm,bottom=2cm]{geometry}
\usepackage[cm-default]{fontspec}
\usepackage{xunicode}

%\tolerance=1000
%\emergencystretch=0.74cm 

\usepackage{polyglossia}
\setdefaultlanguage[spelling=modern]{russian}
\setotherlanguage{english} 
\defaultfontfeatures{Scale=MatchLowercase,Ligatures=TeX}  %% устанавливает поведение шрифтов по умолчанию  
\newfontfamily\cyrillicfont{Linux Libertine} 
\setromanfont[Mapping=tex-text]{Linux Libertine}
\setsansfont[Mapping=tex-text]{Linux Biolinum}
\setmonofont{DejaVu Sans Mono}
%\newfontfamily\cyrillicfont{Liberation Mono} 

%\usepackage{makecell}

%\usepackage{titlesec}
%\newcommand{\sectionbreak}{\clearpage}

%\renewcommand{\thesection}{\Alph{section}}
%\newcount\wd    \wd=\textwidth \multiply\wd by 8 \divide\wd by 17

\usepackage{minted}
\usemintedstyle{friendly}

\usepackage[unicode, pdfborder={0 0 0 0}]{hyperref}

\author{Alaksiej Stankievič}
\title{Домашнее задание}

\begin{document}
\hypersetup{
pdftitle = {Задание 15},
pdfauthor = {Alaksiej Stankievič},
pdfsubject = {домашнее задание}
}% End of hypersetup

\section*{Линейный конгруэнтный генератор псевдослучайных чисел}
Для большинства программ\footnote{К указанному большинству не относятся программы имеющие касательство к криптографии (шифрование, пароли и т.п.)} использующих случайность достаточно не настоящих случайных чисел, а лишь чисел, которые ведут себя похожим образом. Поэтому на практике находят очень широкое применение алгоритмы генерации псевдослучайных чисел. Простейшим и обычно самым быстрым из них является линейный конгруэнтный метод\footnote{Так функция rand() из stdlib.h реализована именно таким образом.}.

Суть метода выражается в нахождение следующего псевдослучайного числа по предыдущему по представленной формуле:
\begin{equation}
 x_{n+1}=(a x_n + c) \bmod m,
\end{equation}
где $x_{n+1}$ новое псевдослучайное число, $x_n$ --- старое, $\bmod$ это сравнимость по модулю\footnote{Которую также называют конгруэнтностью, откуда и название метода}, что практически полностью соответствует нахождению остатка при делении на $m$, а  $1\leq{}a<m$, $0<c<m$ и $m$ некоторые коэффициенты, которые и определяют последовательность псевдослучайных чисел.

Не каждый набор коэффициентов определяет последовательность похожую на случайные числа, так например $a=7$ $c=8$ $m=16$, порождают последовательность чередующихся 0 и 8.

Для хорошей генерации должны быть выполнены следующие правила:
\begin{enumerate}
 \item Числа c и m взаимно простые;
 \item $a - 1$ кратно $p$ для каждого простого $p$, являющегося делителем $m$;
 \item $a - 1$ кратно 4, если $m$ кратно 4.
\end{enumerate}

\subsection*{Статические переменные}
Переменные объявленные с ключевым словом static сохраняют своё значения между вызовами функций.

\section{Задачи}

\begin{enumerate}
 \item Реализовать линейный конгруэнтный генератор псевдослучайных чисел.
   \begin{enumerate}
    \item Используя коэффициенты из википедии, например a=4096, c=150889, m=714025.
    \item Подобрав собственные коэффициенты по правилам, не использую готовые коэффициенты из википедии. 
   \end{enumerate}

 \item В программе объявлен массив целых чисел фиксированной длинны. Пользователь вводит массив, затем массив выводится, затем пользователь вводит число, если такое число есть печатается его индекс, если такого числа нет об этом сообщается пользователю. Функция осуществляющая поиск возвращает первый левый индекс, если таких чисел несколько, и -1 если такого числа нет.
 \item В программе объявлен массив целых чисел фиксированного размера. Пользователь вводит число реально используемой части массива, которая затем заполняется случайными числами. Затем программа выводит максимальны и минимальный элемент в массиве. (Должна быть одна функция для поиска минимума и одна для максимума).
\end{enumerate}



\end{document}
