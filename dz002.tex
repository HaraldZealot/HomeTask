\documentclass[12pt]{article} 
\usepackage[xetex, a4paper, left=2cm, right=2cm, top=2cm,bottom=2cm]{geometry}
\usepackage[cm-default]{fontspec}
\usepackage{xunicode}

%\tolerance=1000
%\emergencystretch=0.74cm 

\usepackage{polyglossia}
\setdefaultlanguage[spelling=modern]{russian}
\setotherlanguage{english} 
\defaultfontfeatures{Scale=MatchLowercase,Ligatures=TeX}  %% устанавливает поведение шрифтов по умолчанию  
\newfontfamily\cyrillicfont{Linux Libertine} 
\setromanfont[Mapping=tex-text]{Linux Libertine}
\setsansfont[Mapping=tex-text]{Linux Biolinum}
\setmonofont{DejaVu Sans Mono}
%\newfontfamily\cyrillicfont{Liberation Mono} 

%\usepackage{makecell}

%\usepackage{titlesec}
%\newcommand{\sectionbreak}{\clearpage}

%\renewcommand{\thesection}{\Alph{section}}
%\newcount\wd    \wd=\textwidth \multiply\wd by 8 \divide\wd by 17

\usepackage{minted}
\usemintedstyle{friendly}

\author{Alaksiej Stankievič}
\title{Домашние задание 2}

\begin{document}
 Написать алгоритмы в виде псевдокода или блок-схемы.
 \begin{enumerate}
  \item Перевод числа из двоичной системы счисления в десятичную.
  \item Перевод числа из десятичной системы счисления в двоичную.
  \item Пользователь вводит число $b$ и число $x$. Перевести число $x$ из десятичной в $b$-ную систему счисления. 
  \item Ряд $1, 1, 2, 3, 5, 8, 13, 21, 34...$ называется числами Фибоначчи. Точное определение чисел Фибоначчи следующее $F_1 =1, F_2=1, F_i=F_{i-1}+F_{i-2}$. Пользователь вводит число $n, n\geq3$, вывести все $n$ чисел Фибоначчи через пробел.
  \item\footnote{Это задание было в дневной группе, но вечерняя тоже может его делать.} Дано два отрезка $a$ и $b$. Найти наибольший отрезок, который целое число раз укладывается в оба отрезка.
  \item\footnote{Это задание было в вечерней группе, но дневная тоже может его делать.} Написать алгоритмы сложения, умножения и деления столбиком на бумаге.
 \end{enumerate}

\end{document}
