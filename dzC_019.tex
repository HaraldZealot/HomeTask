\documentclass[12pt]{article} 
\usepackage[xetex, a4paper, left=2cm, right=2cm, top=2cm,bottom=2cm]{geometry}
\usepackage[cm-default]{fontspec}
\usepackage{xunicode}

%\tolerance=1000
%\emergencystretch=0.74cm 

\usepackage{polyglossia}
\setdefaultlanguage[spelling=modern]{russian}
\setotherlanguage{english} 
\defaultfontfeatures{Scale=MatchLowercase,Ligatures=TeX}  %% устанавливает поведение шрифтов по умолчанию  
\newfontfamily\cyrillicfont{Linux Libertine} 
\setromanfont[Mapping=tex-text]{Linux Libertine}
\setsansfont[Mapping=tex-text]{Linux Biolinum}
\setmonofont{DejaVu Sans Mono}
%\newfontfamily\cyrillicfont{Liberation Mono} 

%\usepackage{makecell}

%\usepackage{titlesec}
%\newcommand{\sectionbreak}{\clearpage}

%\renewcommand{\thesection}{\Alph{section}}
%\newcount\wd    \wd=\textwidth \multiply\wd by 8 \divide\wd by 17

\usepackage{minted}
\usemintedstyle{friendly}
\renewcommand\listingscaption{Код}
\newminted{bash}{frame=lines}
\newminted{c}{frame=leftline}

\usepackage[unicode, pdfborder={0 0 0 0}]{hyperref}

\author{Alaksiej Stankievič}
\title{Домашнее задание}

\begin{document}
\hypersetup{
pdftitle = {Задание 19},
pdfauthor = {Alaksiej Stankievič},
pdfsubject = {домашнее задание}
}% End of hypersetup

\section{Задание}

Реализовать одно из двух на выбор (или оба для получения бонуса).

\begin{enumerate}
 \item Трёхмерные крестики нолики, в кубе $4\times 4 \times 4$. Побеждает игрок выстроивший линию из 4. Куб рисуется по этажам.
 \item Плоские крестики нолики на большом поле (минимум $20 \times 20$). Побеждает игрок выстроивший линию из 5.
\end{enumerate}

\subsection{Совет}
Мне видится удобным для упрощения проверок на выигрыш реализовать функцию проверки одной линии следующей сигнатуры:
\begin{listing}[H]
\begin{center}
\begin{ccode}
int checkLine(int field[][4][4], int length, int i, int j, int k, 
                int vi, int vj, int vk);
\end{ccode}
\end{center}
\caption{Проверка одной линии}
\label{lst:checkline}
\end{listing} 
Функция возвращает 0, если на линии нет победителя, 1 если линия заполнена ноликами, 2 если крестиками. Параметр field это игровое поле (в коде \ref{lst:checkline}, рассмотрен случай трёхмерных крестиков ноликов, в плоских всё похоже). Параметр length это длина линии\footnote{В принципе её можно зашить в код функции, но так получается более масштабируемое (особенно для плоского случая) решение.}. Параметры i, j, k это индексы начальной ячейки линии. А параметры vi, vj, vk определяют куда линия направлена\footnote{С математической точки зрения это не что иное, как координаты направляющего вектора, отсюда и название параметров}, они могут принимать только три значения -1, 0 и 1. Например, рассмотрим случай когда $(i, j, k) = (0, 0, 0)$ и $(vi, vj, vk) = (1, 1, 0)$, тогда первая ячейка линии $(0, 0, 0)$, вторая ячейка получается из первой прибавлением соответствующих компонент $v$, а именно в нашем случае $(1, 1, 0)$, третья --- прибавлением компонент $v$, что даёт $(2, 2, 0)$ и четвёртая в итоге $(3, 3, 0)$, то есть мы получили диагональную линию на этаже с индексом 0. Если $(vi, vj, vk) = (1, 0, 0)$, то получится вертикальная линия на этаже с индексом 0, а если $(vi, vj, vk) = (1, 1, 1)$, то получится диагональ через весь куб.

\end{document}
