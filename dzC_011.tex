\documentclass[12pt]{article} 
\usepackage[xetex, a4paper, left=2cm, right=2cm, top=2cm,bottom=2cm]{geometry}
\usepackage[cm-default]{fontspec}
\usepackage{xunicode}

%\tolerance=1000
%\emergencystretch=0.74cm 

\usepackage{polyglossia}
\setdefaultlanguage[spelling=modern]{russian}
\setotherlanguage{english} 
\defaultfontfeatures{Scale=MatchLowercase,Ligatures=TeX}  %% устанавливает поведение шрифтов по умолчанию  
\newfontfamily\cyrillicfont{Linux Libertine} 
\setromanfont[Mapping=tex-text]{Linux Libertine}
\setsansfont[Mapping=tex-text]{Linux Biolinum}
\setmonofont{DejaVu Sans Mono}
%\newfontfamily\cyrillicfont{Liberation Mono} 

%\usepackage{makecell}

%\usepackage{titlesec}
%\newcommand{\sectionbreak}{\clearpage}

%\renewcommand{\thesection}{\Alph{section}}
%\newcount\wd    \wd=\textwidth \multiply\wd by 8 \divide\wd by 17

\usepackage{minted}
\usemintedstyle{friendly}

\usepackage[unicode, pdfborder={0 0 0 0}]{hyperref}

\author{Alaksiej Stankievič}
\title{Домашнее задание}

\begin{document}
\hypersetup{
pdftitle = {Задание 8},
pdfauthor = {Alaksiej Stankievič},
pdfsubject = {домашнее задание}
}% End of hypersetup


 Запрограммировать на языке С.
 \begin{enumerate}
  \item Пользователь вводит два натуральных числа $m, n$. Заполнить поле $m\times{}n$ звёздочками в шахматном порядке.
  \item Саша и Вася положили в банк одинаковую сумму: 1000\$. Саша положил под простой\footnote{Процент всегда 
  начисляется от изначальной суммы на вкладе.} процент, ставка равна 6\%, а Вася под сложный\footnote{Процент 
начисляется от текущей суммы на вкладе.}, ставка --- 4\%. Процент начисляется ежеквартально (каждые три месяца). Найдите 
через какое время Вася будет иметь на счету большую сумму. Выведете  историю изменения суммы на счёте, до момента, когда 
Вася станет богаче, и ещё на протяжении трёх кварталов. Счёт выводить с точностью до центов, но сами деньги хранить 
внутри программы как целые числа.
  \item Проверить является ли введённое пользователем число палиндромом\footnote{Палиндром это число или фраза, 
  которые одинаково читаются в обоих направлениях. Например, число 121 палиндром, а 112 --- нет.}.
  \item Перевести введённое пользователем число в 12-ичную систему счисления.
 \end{enumerate}

\end{document}
