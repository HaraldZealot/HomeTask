\documentclass[12pt]{article} 
\usepackage[xetex, a4paper, left=2cm, right=2cm, top=2cm,bottom=2cm]{geometry}
\usepackage[cm-default]{fontspec}
\usepackage{xunicode}

%\tolerance=1000
%\emergencystretch=0.74cm 

\usepackage{polyglossia}
\setdefaultlanguage[spelling=modern]{russian}
\setotherlanguage{english} 
\defaultfontfeatures{Scale=MatchLowercase,Ligatures=TeX}  %% устанавливает поведение шрифтов по умолчанию  
\newfontfamily\cyrillicfont{Linux Libertine} 
\setromanfont[Mapping=tex-text]{Linux Libertine}
\setsansfont[Mapping=tex-text]{Linux Biolinum}
\setmonofont{DejaVu Sans Mono}
%\newfontfamily\cyrillicfont{Liberation Mono} 

%\usepackage{makecell}

%\usepackage{titlesec}
%\newcommand{\sectionbreak}{\clearpage}

%\renewcommand{\thesection}{\Alph{section}}
%\newcount\wd    \wd=\textwidth \multiply\wd by 8 \divide\wd by 17

\usepackage{minted}
\usemintedstyle{friendly}
\renewcommand\listingscaption{Код}
\newminted{bash}{frame=lines}
\newminted{c}{frame=leftline}

\usepackage[unicode, pdfborder={0 0 0 0}]{hyperref}

\author{Alaksiej Stankievič}
\title{Домашнее задание}

\begin{document}
\hypersetup{
pdftitle = {Задание 22},
pdfauthor = {Alaksiej Stankievič},
pdfsubject = {домашнее задание}
}% End of hypersetup

\section{Задание}


\begin{enumerate}
 \item Написать программу работы со смешанными дробями (целая часть, числитель, знаменатель). Должны быть функции
ввода, вывода, арифметических операций и преоброзования в double.
 \item Написать программу работы с комплексными числами. Должны быть функции ввода, вывода, арифметических операций и
комплексного сопряжения.
 \item (бонус) Написать программу работы с кватэрнионами ($q=w+ix+jy+kz$, где $w$, $x$, $y$ и $z$ действительные числа,
а $i$, $j$ и $k$ мнимые единицы со следующими правилами: $i^2=j^2=k^2=-1$, $ij=k$, $jk=i$ и $ki=j$). Должны быть функции
ввода, вывода и арифметических операций.
 \item Написать программу для работы со структурой время (часы, минуты, секунды). Должны быть функции
ввода, вывода, прибавления к моменту времени секунд, разница между моментами времени в секундах.
 \item Написать программу для работы со структурой дата (год, месяц, день). Должны быть функции
ввода, вывода, прибавления к дате количества дней, разница между датами в днях. Учесть устройство григорианского
календаря.
 \item Реализовать простую базу данных (например горы: название горы, название горного масива, высота, крутизна,
наличие ледника). Она представленна в виде массива структур. Должно быть текстовое меню. Возможность загружать и
сохранять базу данных в бинарном виде, добавлять и удалять записи из масива, выводить базу данных, сортировать по
любому из полей в обоих направлениях. Самостоятельно реализовать алгоритм сортировки (один из быстрых) с исользованием
указателей на функции.
 \end{enumerate}



\end{document}
