\documentclass[12pt]{article} 
\usepackage[xetex, a4paper, left=2cm, right=2cm, top=2cm,bottom=2cm]{geometry}
\usepackage[cm-default]{fontspec}
\usepackage{xunicode}

%\tolerance=1000
%\emergencystretch=0.74cm 

\usepackage{polyglossia}
\setdefaultlanguage[spelling=modern]{russian}
\setotherlanguage{english} 
\defaultfontfeatures{Scale=MatchLowercase,Ligatures=TeX}  %% устанавливает поведение шрифтов по умолчанию  
\newfontfamily\cyrillicfont{Linux Libertine} 
\setromanfont[Mapping=tex-text]{Linux Libertine}
\setsansfont[Mapping=tex-text]{Linux Biolinum}
\setmonofont{DejaVu Sans Mono}
%\newfontfamily\cyrillicfont{Liberation Mono} 

%\usepackage{makecell}

%\usepackage{titlesec}
%\newcommand{\sectionbreak}{\clearpage}

%\renewcommand{\thesection}{\Alph{section}}
%\newcount\wd    \wd=\textwidth \multiply\wd by 8 \divide\wd by 17

\usepackage{minted}
\usemintedstyle{friendly}

\usepackage[unicode, pdfborder={0 0 0 0}]{hyperref}

\author{Alaksiej Stankievič}
\title{Домашнее задание}

\begin{document}
\hypersetup{
pdftitle = {Задание 7},
pdfauthor = {Alaksiej Stankievič},
pdfsubject = {домашнее задание}
}% End of hypersetup


 Запрограммировать на языке С.
 \begin{enumerate}
  \item Пользователь вводит натуральное число $x\geq{}2$. Проверить является ли оно простым\footnote{Число называется простым если оно нацело делится только на себе и на 1}.
 \item Разложить натуральное число $x\geq{}2$ на простые сомножители  
 \footnote{\textit{Примеры}: 12 = 2 * 2 * 3,  144 = 2 * 2 * 2 * 2 * 3 * 3, 121 = 11 * 11, 17 = 17. Можно, также, писать в виде 144 = (2\textasciicircum4) *\\ (3\textasciicircum2), 17 = (17\textasciicircum1)}
  \item Пользователь вводит нечётное число $n$ выведите ромбик символом \verb|#| диагонали которого равны $n$.
  \item Реализовать двоичное возведение в целую степень дробного числа.
  \item Найти сумму цифр числа.
  \item (Сложная на бонус) По любому натуральному число построить следующее натуральное число с той же суммой цифр.
 \end{enumerate}

\end{document}
