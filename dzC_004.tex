\documentclass[12pt]{article} 
\usepackage[xetex, a4paper, left=2cm, right=2cm, top=2cm,bottom=2cm]{geometry}
\usepackage[cm-default]{fontspec}
\usepackage{xunicode}

%\tolerance=1000
%\emergencystretch=0.74cm 

\usepackage{polyglossia}
\setdefaultlanguage[spelling=modern]{russian}
\setotherlanguage{english} 
\defaultfontfeatures{Scale=MatchLowercase,Ligatures=TeX}  %% устанавливает поведение шрифтов по умолчанию  
\newfontfamily\cyrillicfont{Linux Libertine} 
\setromanfont[Mapping=tex-text]{Linux Libertine}
\setsansfont[Mapping=tex-text]{Linux Biolinum}
\setmonofont{DejaVu Sans Mono}
%\newfontfamily\cyrillicfont{Liberation Mono} 

%\usepackage{makecell}

%\usepackage{titlesec}
%\newcommand{\sectionbreak}{\clearpage}

%\renewcommand{\thesection}{\Alph{section}}
%\newcount\wd    \wd=\textwidth \multiply\wd by 8 \divide\wd by 17

\usepackage{minted}
\usemintedstyle{friendly}

\author{Alaksiej Stankievič}
\title{Домашнее задание 4}

\begin{document}
 Запрограммировать на языке С. Работать только с целыми числами. Математических функций не использовать.
 \begin{enumerate}
  \item Вывести фразу "Hello world!" 137 раз.
  \item Написать через пробел числа от 1 до $n$, $n$ вводит пользователь.
  \item Написать в столбик таблицу кубов первых 17 нечётных чисел.
  \item Пользователь вводит число $n\geq0$, а затем $n$ чисел. Вывести произведения этих чисел.
  \item Пользователь вводит число $x$ и степень $p\geq0$. Вывести $x^p$.
  \item Пользователь вводит число $L>0$ и число $x\leq{}L$. Найти и вывести степень $p$ такую, что 
  $x^p$ наиболее близко снизу к $L$. (Конкретнее: $x^p\leq{}L$, а $x^{p+1}>L$).
 \end{enumerate}
 
 Ответьте на вопрос: в каких случаях следует применять цикл for, а в каких while?
  

\end{document}
