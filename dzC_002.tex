\documentclass[12pt]{article} 
\usepackage[xetex, a4paper, left=2cm, right=2cm, top=2cm,bottom=2cm]{geometry}
\usepackage[cm-default]{fontspec}
\usepackage{xunicode}

%\tolerance=1000
%\emergencystretch=0.74cm 

\usepackage{polyglossia}
\setdefaultlanguage[spelling=modern]{russian}
\setotherlanguage{english} 
\defaultfontfeatures{Scale=MatchLowercase,Ligatures=TeX}  %% устанавливает поведение шрифтов по умолчанию  
\newfontfamily\cyrillicfont{Linux Libertine} 
\setromanfont[Mapping=tex-text]{Linux Libertine}
\setsansfont[Mapping=tex-text]{Linux Biolinum}
\setmonofont{DejaVu Sans Mono}
%\newfontfamily\cyrillicfont{Liberation Mono} 

%\usepackage{makecell}

%\usepackage{titlesec}
%\newcommand{\sectionbreak}{\clearpage}

%\renewcommand{\thesection}{\Alph{section}}
%\newcount\wd    \wd=\textwidth \multiply\wd by 8 \divide\wd by 17

\usepackage{minted}
\usemintedstyle{friendly}

\author{Alaksiej Stankievič}
\title{Домашние задание 2}

\begin{document}
\section{Системы счисления}
\begin{enumerate}
 \item Написать все числа подряд (лучше в столбик) в двоичной системе счисления от 0 до 255 включительно
 \item Перевести следующие числа в 10-чную систему счисления 
 \begin{enumerate}
    \item $101111011_2$, $111111111_2$, $00000001_2$, $1011011101111_2$
    \item $12021_3$
    \item $7135_8$
    \item $11_{16}$, $A7_{16}$, $CAFE_{16}$
 \end{enumerate}
 
 \item Перевести следующие числа в 2-чную систему счисления
 \begin{enumerate}
  \item $345_{10}$, $17399_{10}$, $255_{10}$
  \item $123_8$
  \item $FF_{16}$, $BA_{16}$, $CAFE_{16}$
 \end{enumerate}
 \item Перевести следующие числа в 16-чную систему счисления
 \begin{enumerate}
  \item $101111011_2$, $1011011101111_2$
  \item $7135_8$
  \item $456239_{10}$
 \end{enumerate}

 \item сложить напрямую в двоичной системе счисления 
 \begin{enumerate}
  \item $1011110111101_2$ и $1110111_2$
  \item $10111111101101_2$ и $10111111101101_2$
 \end{enumerate}
 \item умножить напрямую в двоичной системе счисления $1101110101_2$ и $1010111010_2$
 \item написать таблицу умножения 16-ной системы счисления
\end{enumerate}

\section{Алгоритмы}
 Написать алгоритмы в виде псевдокода или блок-схемы.
 \begin{enumerate}
  \item Перевод числа из двоичной системы счисления в десятичную.
  \item Перевод числа из десятичной системы счисления в двоичную.
  \item Пользователь вводит число $b$ и число $x$. Перевести число $x$ из десятичной в $b$-ную систему счисления. 
  \item Ряд $1, 1, 2, 3, 5, 8, 13, 21, 34...$ называется числами Фибоначчи. Точное определение чисел Фибоначчи следующее $F_1 =1, F_2=1, F_i=F_{i-1}+F_{i-2}$. Пользователь вводит число $n, n\geq3$, вывести все $n$ чисел Фибоначчи через пробел.
  \item Дано два отрезка $a$ и $b$. Найти наибольший отрезок, который целое число раз укладывается в оба отрезка.
  \item Написать алгоритмы сложения, умножения и деления столбиком на бумаге.
 \end{enumerate}

\end{document}
