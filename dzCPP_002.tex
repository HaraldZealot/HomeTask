\documentclass[12pt]{article} 
\usepackage[xetex, a4paper, left=2cm, right=2cm, top=2cm,bottom=2cm]{geometry}
\usepackage[cm-default]{fontspec}
\usepackage{xunicode}

%\tolerance=1000
%\emergencystretch=0.74cm 

\usepackage{polyglossia}
\setdefaultlanguage[spelling=modern]{russian}
\setotherlanguage{english} 
\defaultfontfeatures{Scale=MatchLowercase,Ligatures=TeX}  %% устанавливает поведение шрифтов по умолчанию  
\newfontfamily\cyrillicfont{Linux Libertine} 
\setromanfont[Mapping=tex-text]{Linux Libertine}
\setsansfont[Mapping=tex-text]{Linux Biolinum}
\setmonofont{DejaVu Sans Mono}
%\newfontfamily\cyrillicfont{Liberation Mono} 

%\usepackage{makecell}

%\usepackage{titlesec}
%\newcommand{\sectionbreak}{\clearpage}

%\renewcommand{\thesection}{\Alph{section}}
%\newcount\wd    \wd=\textwidth \multiply\wd by 8 \divide\wd by 17

\usepackage{minted}
\usemintedstyle{friendly}
\renewcommand\listingscaption{Код}
\newminted{bash}{frame=lines}
\newminted{c}{frame=leftline}

\usepackage[unicode, pdfborder={0 0 0 0}]{hyperref}

\author{Alaksiej Stankievič}
\title{Домашнее задание}

\begin{document}
\hypersetup{
pdftitle = {Задание 2},
pdfauthor = {Alaksiej Stankievič},
pdfsubject = {домашнее задание}
}% End of hypersetup

\section*{Матричное умножение}
Если есть матрица $A$ размерности $m\times{}n$ с элементами $a_{ij}$ и матрица $B$ размерности $n\times{}p$ с 
элементами $b_{jk}$, то в результате получается матрица $C$ размерности $m\times{}p$ элементы которой считаются по 
следующей формуле:
\begin{equation}
 c_{ik}=\sum_{j=1}^{n}a_{ij}b_{jk}
\end{equation}


\section*{Задание}

Реализовать класс прямоугольных вещественных матриц размерности $m\times{}n$. В обязательном порядке реализовать всю 
великую 6, а также операторы сложения, вычитания и умножения матриц соответствующего размера (кидайте исключение или 
возвращайте невалидную матрицу если операция невозможна.), реализуйте функции ввода и вывода матрицы.



\end{document}
