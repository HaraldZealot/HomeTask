\documentclass[12pt]{article} 
\usepackage[xetex, a4paper, left=2cm, right=2cm, top=2cm,bottom=2cm]{geometry}
\usepackage[cm-default]{fontspec}
\usepackage{xunicode}

%\tolerance=1000
%\emergencystretch=0.74cm 

\usepackage{polyglossia}
\setdefaultlanguage[spelling=modern]{russian}
\setotherlanguage{english} 
\defaultfontfeatures{Scale=MatchLowercase,Ligatures=TeX}  %% устанавливает поведение шрифтов по умолчанию  
\newfontfamily\cyrillicfont{Linux Libertine} 
\setromanfont[Mapping=tex-text]{Linux Libertine}
\setsansfont[Mapping=tex-text]{Linux Biolinum}
\setmonofont{DejaVu Sans Mono}
%\newfontfamily\cyrillicfont{Liberation Mono} 

%\usepackage{makecell}

%\usepackage{titlesec}
%\newcommand{\sectionbreak}{\clearpage}

%\renewcommand{\thesection}{\Alph{section}}
%\newcount\wd    \wd=\textwidth \multiply\wd by 8 \divide\wd by 17

\usepackage{minted}
\usemintedstyle{friendly}

\author{Alaksiej Stankievič}
\title{Домашние задание 3}

\begin{document}
 \section{Запрограммировать на языке C}
  Написать программу, которая предлагает пользователю ввести два числа, а затем выводит 5 арифметических операций в следующем виде:
  \begin{verbatim}
7 + 5 = 12
7 - 5 = 2
7 * 5 = 35
7 / 5 = 1
7 % 5 = 2
  \end{verbatim}
 \section{Поупражняться в тетради}
  \begin{itemize}
   \item Написать таблицу сложения и умножения в двоичной и шестнадцатеричной системе.
   \item Сложить столбиком числа содержащие не менее 5 цифр в двоичной и шестнадцатеричной системе
   \item Умножить столбиком числа содержащие не менее 5 цифр в двоичной и шестнадцатеричной системе
   \item Разделить столбиком числа в двоичной системе.
  \end{itemize}


\end{document}
