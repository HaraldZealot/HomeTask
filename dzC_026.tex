\documentclass[12pt]{article} 
\usepackage[xetex, a4paper, left=2cm, right=2cm, top=2cm,bottom=2cm]{geometry}
\usepackage[cm-default]{fontspec}
\usepackage{xunicode}

%\tolerance=1000
%\emergencystretch=0.74cm 

\usepackage{polyglossia}
\setdefaultlanguage[spelling=modern]{russian}
\setotherlanguage{english} 
\defaultfontfeatures{Scale=MatchLowercase,Ligatures=TeX}  %% устанавливает поведение шрифтов по умолчанию  
\newfontfamily\cyrillicfont{Linux Libertine} 
\setromanfont[Mapping=tex-text]{Linux Libertine}
\setsansfont[Mapping=tex-text]{Linux Biolinum}
\setmonofont{DejaVu Sans Mono}
%\newfontfamily\cyrillicfont{Liberation Mono} 

%\usepackage{makecell}

%\usepackage{titlesec}
%\newcommand{\sectionbreak}{\clearpage}

%\renewcommand{\thesection}{\Alph{section}}
%\newcount\wd    \wd=\textwidth \multiply\wd by 8 \divide\wd by 17

\usepackage{minted}
\usemintedstyle{friendly}
\renewcommand\listingscaption{Код}
\newminted{bash}{frame=lines}
\newminted{c}{frame=leftline}

\usepackage[unicode, pdfborder={0 0 0 0}]{hyperref}

\author{Alaksiej Stankievič}
\title{Домашнее задание}

\begin{document}
\hypersetup{
pdftitle = {Задание 26},
pdfauthor = {Alaksiej Stankievič},
pdfsubject = {домашнее задание}
}% End of hypersetup

\section{Задание}


\begin{enumerate}
 \item Разобраться с устройством UNICODE,  и в частности с устройством 
кодировки UTF-8. Реализовать две следующих функции получающих строку в 
кодировке UTF-8:
\begin{enumerate}
 \item Подсчёта кодовых точек в строке (реальных символов). 
 \item Получения массива int'ов, в котором записаны коды кодовых точек.
\end{enumerate}

 \item С помощью библиотеки SDL2 написать программу которая рисует фигуры 
Лиссажу. Фигура Лиссажу это кривая которая задаётся параметричесаким уравнением 
\begin{equation}
 \left\{
 \begin{array}{l}x=cos(mt),\\y=sin(nt). \end{array}                        
               \right. ,
\end{equation}
где $m$ и $n$ натуральные числа, которые можно изменять в программе с помощью 
клавиш направления.

\end{enumerate}




\end{document}
