\documentclass[12pt]{article} 
\usepackage[xetex, a4paper, left=2cm, right=2cm, top=2cm,bottom=2cm]{geometry}
\usepackage[cm-default]{fontspec}
\usepackage{xunicode}

%\tolerance=1000
%\emergencystretch=0.74cm 

\usepackage{polyglossia}
\setdefaultlanguage[spelling=modern]{russian}
\setotherlanguage{english} 
\defaultfontfeatures{Scale=MatchLowercase,Ligatures=TeX}  %% устанавливает поведение шрифтов по умолчанию  
\newfontfamily\cyrillicfont{Linux Libertine} 
\setromanfont[Mapping=tex-text]{Linux Libertine}
\setsansfont[Mapping=tex-text]{Linux Biolinum}
\setmonofont{DejaVu Sans Mono}
%\newfontfamily\cyrillicfont{Liberation Mono} 

%\usepackage{makecell}

%\usepackage{titlesec}
%\newcommand{\sectionbreak}{\clearpage}

%\renewcommand{\thesection}{\Alph{section}}
%\newcount\wd    \wd=\textwidth \multiply\wd by 8 \divide\wd by 17

\usepackage{minted}
\usemintedstyle{friendly}
\renewcommand\listingscaption{Код}
\newminted{bash}{frame=lines}
\newminted{c}{frame=leftline}

\usepackage[unicode, pdfborder={0 0 0 0}]{hyperref}

\author{Alaksiej Stankievič}
\title{Домашнее задание}

\begin{document}
\hypersetup{
pdftitle = {Задание 6},
pdfauthor = {Alaksiej Stankievič},
pdfsubject = {домашнее задание}
}% End of hypersetup


 Запрограммировать на языке С.
 
 Вызов функции \verb|system("cls");| на windows или \verb|system("clear");| на  linux очищает всю консоль.
 \begin{enumerate}
  \item Реализовать алгоритм Евклида нахождения НОД\footnote{наибольший общий делитель, по-английски great common 
devisor (gcd).} двух чисел.
  \item Найти НОК\footnote{наименьшее общее кратное, по-английски least common multiply (lcm).} двух 
чисел\footnote{\textit{Указание:} воспользуйтесь НОД}.
  \item Покажите к чему стремится отношение соседних чисел Фибоначчи. Для этого выведите в столбик отношение 
последующего к предыдущему и предыдущего к последующему для первых 40 чисел Фибоначчи.

  \item Пользователь вводит натуральные числа $w$ и $h$. Напечатать звёздочками прямоугольник введённой ширины и высоты.
  \begin{listing}[H]
\begin{center}
\begin{minted}[frame=lines]{text}
******
******
******
******
\end{minted}
\end{center}

\caption{Прямоугольник с w равной 6 и h равной 4.}
\label{lst:exampleSqure}
\end{listing}
  \item Пользователь вводит ширину $w\geq{}1$ и высоту $h\geq{}1$, выведите соответствующую рамку символом \verb|#|.
    \begin{listing}[H]
\begin{center}
\begin{minted}[frame=lines]{text}
######
#    #
#    #
######
\end{minted}
\end{center}

\caption{Рамка с w равной 6 и h равной 4.}
\label{lst:exampleFrame}
\end{listing}

  
 \end{enumerate}

\end{document}
