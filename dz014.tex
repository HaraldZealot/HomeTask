\documentclass[12pt]{article} 
\usepackage[xetex, a4paper, left=2cm, right=2cm, top=2cm,bottom=2cm]{geometry}
\usepackage[cm-default]{fontspec}
\usepackage{xunicode}

%\tolerance=1000
%\emergencystretch=0.74cm 

\usepackage{polyglossia}
\setdefaultlanguage[spelling=modern]{russian}
\setotherlanguage{english} 
\defaultfontfeatures{Scale=MatchLowercase,Ligatures=TeX}  %% устанавливает поведение шрифтов по умолчанию  
\newfontfamily\cyrillicfont{Linux Libertine} 
\setromanfont[Mapping=tex-text]{Linux Libertine}
\setsansfont[Mapping=tex-text]{Linux Biolinum}
\setmonofont{DejaVu Sans Mono}
%\newfontfamily\cyrillicfont{Liberation Mono} 

%\usepackage{makecell}

%\usepackage{titlesec}
%\newcommand{\sectionbreak}{\clearpage}

%\renewcommand{\thesection}{\Alph{section}}
%\newcount\wd    \wd=\textwidth \multiply\wd by 8 \divide\wd by 17

\usepackage{minted}
\usemintedstyle{friendly}

\usepackage[unicode, pdfborder={0 0 0 0}]{hyperref}

\author{Alaksiej Stankievič}
\title{Домашнее задание}

\begin{document}
\hypersetup{
pdftitle = {Задание 14},
pdfauthor = {Alaksiej Stankievič},
pdfsubject = {домашнее задание}
}% End of hypersetup

Во всех задачах, если специально не оговорено иное, необходимо разбить программу на несколько файлов.

\section{Задачи}

\begin{enumerate}
 \item Создайте папку для собственных модулей в вашем домашнем репозитории\footnote{Например, дайте ей имя mylib или mymodules.}. В ней разместите заголовочный файл и файл реализации расширенной математики в которых реализованы функции нахождения НОД и НОК. В папке домашнего задания 14, создайте программу, которая реализует задачу 13.3 использую функцию gcd из вашего модуля\footnote{У вас будет в этой программе только main.c в котором будет примерно следующее \#include "../../mymodules/mathx.h". Не забудьте добавить файлы из вашего модуля в проект}.
 \item Написать функцию перевода произвольного числа в произвольную систему счисления с основанием от 2 до 36\footnote{Чтобы хватило латинских букв для цифр.}.
 \item Напишите игру в кости между человеком и компьютером. Они по очереди кидают по два кубика. Побеждает тот у кого сумма больше. Кубики рисуйте псевдографикой (см. рисунки \ref{lst:cube3} и \ref{lst:cube6} в качестве примера).
\end{enumerate}

\begin{listing}[H]
\begin{center}
\begin{minted}[frame=single]{text}
+---+
|O  |
| O |
|  O|
+---+
\end{minted}
\end{center}
\caption{Кубик 3}
\label{lst:cube3}
\end{listing}

\begin{listing}[H]
\begin{center}
\begin{minted}[frame=single]{text}
+---+
|OOO|
|   |
|OOO|
+---+
\end{minted}
\end{center}
\caption{Кубик 6}
\label{lst:cube6}
\end{listing}

\end{document}
