\documentclass[12pt]{article} 
\usepackage[xetex, a4paper, left=2cm, right=2cm, top=2cm,bottom=2cm]{geometry}
\usepackage[cm-default]{fontspec}
\usepackage{xunicode}

%\tolerance=1000
%\emergencystretch=0.74cm 

\usepackage{polyglossia}
\setdefaultlanguage[spelling=modern]{russian}
\setotherlanguage{english} 
\defaultfontfeatures{Scale=MatchLowercase,Ligatures=TeX}  %% устанавливает поведение шрифтов по умолчанию  
\newfontfamily\cyrillicfont{Linux Libertine} 
\setromanfont[Mapping=tex-text]{Linux Libertine}
\setsansfont[Mapping=tex-text]{Linux Biolinum}
\setmonofont{DejaVu Sans Mono}
%\newfontfamily\cyrillicfont{Liberation Mono} 

%\usepackage{makecell}

%\usepackage{titlesec}
%\newcommand{\sectionbreak}{\clearpage}

%\renewcommand{\thesection}{\Alph{section}}
%\newcount\wd    \wd=\textwidth \multiply\wd by 8 \divide\wd by 17

\usepackage{minted}
\usemintedstyle{friendly}
\renewcommand\listingscaption{Код}
\newminted{bash}{frame=lines}
\newminted{c}{frame=leftline}

\usepackage[unicode, pdfborder={0 0 0 0}]{hyperref}

\author{Alaksiej Stankievič}
\title{Домашнее задание}

\begin{document}
\hypersetup{
pdftitle = {Задание 18},
pdfauthor = {Alaksiej Stankievič},
pdfsubject = {домашнее задание}
}% End of hypersetup

\section{Стек}
Стек --- это абстрактная структура данных поддерживающая протокол LIFO (last input first output --- последним зашёл, первым вышел). Абстрактность в определение означает, что нас не интересует внутренняя реализация, достаточно чтобы нам был предоставлен интерфейс. Интерфейс стека содержит в обязательном порядке два действия:
\begin{enumerate}
 \item поместить данные в конец стека (push),
 \item извлечь данные из стека (pop).
\end{enumerate}
Часто эти действия реализуются в виде соответствующих функций, но мы ещё не знаем механизм передачи по указателю, чтобы грамотно их записать, поэтому пока мы будем писать прямым кодом и копипастом. Также часто функциональность по работе со стеком содержит действия по просмотру вершины без извлечения, проверки на пустоту и полной очистки стека.

Как уже было указано внутренняя реализация стека может быть различной. Одной из таких является реализация с помощью массива, её мы рассмотрим подробнее.

Пусть у нас есть массив, который мы обозначим как stack, и переменная, которую назовём top (то есть верхушка или вершина стека). Изначально top имеет значение -1, а массив stack хоть и содержит информацию, но считается пустым.
Тогда операция push выглядит как:
\begin{listing}[H]
\begin{center}
\begin{ccode}
++top;
stack[top] = input;
\end{ccode}
\end{center}
\caption{Развёрнутая версия push}
\label{lst:pushext}
\end{listing}
Что можно сократить до одной строки. 
\begin{listing}[H]
\begin{center}
\begin{ccode}
stack[++top] = input;
\end{ccode}
\end{center}
\caption{Однострочная версия push}
\label{lst:pushsingle}
\end{listing}

Аналогично выглядит и операции pop в развёрнутом (код~\ref{lst:popext}) и однострочном виде (код~\ref{lst:popsingle}).
\begin{listing}[H]
\begin{center}
\begin{ccode}
output = stack[top];
--top;
\end{ccode}
\end{center}
\caption{Развёрнутая версия pop}
\label{lst:popext}
\end{listing}
\begin{listing}[H]
\begin{center}
\begin{ccode}
output = stack[top--];
\end{ccode}
\end{center}
\caption{Однострочная версия pop}
\label{lst:popsingle}
\end{listing}

\section{Стекплетрон}
Стекплетрон --- это машина построенная на базе Симлетрона, она имеет стек который является особым образом организованным участком памяти. 

Модифицируем машину Симплетрон следующим образом: добавим два новых регистра, регистр смещения (bias) и регистр вершины стека (top). Изменим смысл команд всех команд ввода-вывода, загрузки и арифметики\footnote{Измения не затрагивают команды начинающиеся на 4 --- команды изменения управления}, теперь фактический адрес будет вычисляться по формуле address + bias. Изначально top равен размеру памяти (100, это соответствует -1 на стандартном стеке), а bias --- 0. 
Также снимем ограничение, что машинные слова содержащие команды должны быть только положительным. Если машинное слово отрицательное, то это расшифровывается так, что первые две цифры код команды, а последние две цифры адрес с которым работает команда, но отрицательный, например -3003, означает команду 30 (сложение) по адресу -03. В итоге получается, например, следующая процедура:
\begin{verbatim}
+2003
-3201
\end{verbatim}
Загружаем в аккумулятор по относительному адресу +03\footnote{По нашему соглашению о вызовах это второй параметр функции.} accumulator = memory[address + bias], а затем делим аккумулятор на значение по относительному адресу -01\footnote{По нашему соглашению о вызовах это первая локальная переменная функции.} accumulator /= memory[address + bias]. Если, к примеру, bias был 91, то для предыдущих команд мы получим абсолютные адреса 94 и 90 соответственно.

Введём следующие новые команды:
\begin{itemize}
 \item LOADTOP = 22 --- загружает в аккумулятор значение регистра top,
 \item STORETOP = 23 --- сохраняет значение аккумулятора в регистре top,
 \item LOADBIAS = 24 --- загружает в аккумулятор значение регистра bias,
 \item STOREBIAS =25 --- сохраняет значение аккумулятора в регистре bias,
 \item PUSH = 26 --- помещает значение аккумулятора в стек,
 \item POP = 27 --- извлекает из стека значение и помещает в аккумулятор,

 \item LITERAL = 34 --- часть машинного слова являющаяся адресом напрямую помещается в аккумулятор,

 \item CALL = 44 --- текущее значение регистра инструкций помещается в стек, управление переходит по указанному адресу,
 \item RETURN = 45 --- управление передаётся по адресу хранящемуся в стеке (или по следующему адресу, зависит от особенностей реализации стандартного перехода на следующую инструкцию), из стека выталкивается некоторое число машинных слов, количество представлено в поле адреса (например, +4504 выталкивает 4 параметра). 
\end{itemize}

Для реализации функций будем придерживаться следующих соглашений о вызове и возврате.
\subsection{Вызов функции}
Пункты 1 -- 3 осуществляет код на вызывающей стороне, а пункты 4-5 на вызываемой.
\begin{enumerate}
 \item Если у функции есть возвращаемый аргумент, то выполнить PUSH\footnote{Этим мы выделяем память под будущий результат.}.
 \item Параметры функции помещаются в стек командой PUSH в обратном порядке.
 \item Вызывается код функции командой CALL  по соответствующему адресу.
 \item Значение регистра bias помещается в стек. 
 \item В регистр bias помещается текущее значение регистра top.
\end{enumerate}

Если функции нужны локальные переменные, то память для них резервируется командой PUSH.

В принятом соглашении о вызове функции параметры начинаются по относительному адресу +02.

\subsection{Завершение функции}
Пункты 1 -- 3 осуществляет код функции, а пункт 4 на вызывавший функцию код.
\begin{enumerate}
 \item Командой POP уничтожить все локальные переменные.
 \item Извлечь из стека старое сохранённое значение регистра bias и поместить его в этот регистр.
 \item Вернуться из функции операцией RETURN, в качестве адреса указать ей число входных параметров функции.
 \item Результат функции находится на вершине стека, использовать его или с помощью команды POP или через косвенную адресацию.
\end{enumerate}



\section{Задание}

\begin{enumerate}
 \item Реализовать виртуальную машину Симплетрон, так как она описана у Дейтлов. Написать три собственные программы на ЯМС. 
 \item Реализовать виртуальную машину Стекплетрон, с возможностью вызова функций. Написать три собственные программы на ЯМС, одну с рекурсивным вызовом функций. 
 \item Реализовать стандартные крестики нолики.
\end{enumerate}

\end{document}
